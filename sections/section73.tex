\section{Elementi finiti. Funzioni di forma.}

L'ipotesi alla base del metodo agli elementi finiti è che sia possibile trovare lo spostamento di qualunque punto dell'elemento della maglia conoscendo lo spostamento dei nodi. Per questo serve scrivere la matrice di rigidezza di un elemento completamente generico.

Per un elemento della mesh $j$, i suoi nodi vengono espressi dall'indice $i$. Lo spostamento di uno qualsiasi dei suoi punti si può esprimere in funzione dello spostamento dei nodi attraverso una funzione di forma.
\begin{equation*}
    \left\{u \right\}_j = \begin{Bmatrix}
        u_x\\u_y\\u_z 
    \end{Bmatrix}= [H(x,y,z)] \,\left\{u_i \right\}_j
\end{equation*}

L'unica approssimazione che non ha una natura numerica risiede nell'ipotesi di base, che si traduce nell'uso di funzioni di forma. Serve quindi, attraverso adeguate funzioni di forma $H$, riuscire a rappresentare correttamente il campo di spostamenti e di deformazioni. \\Per questo devono avere le seguenti caratteristiche:
\begin{enumerate}
    \item Congruenza interna. Le funzioni devono essere continue, derivabili e con derivata continua (evitando compenetrazioni e lacerazioni).\\
    \item Congruenza esterna. Non da requisiti su $H$ ma prescrive di eliminare righe e colonne corrispondenti agli spostamenti vincolati.\\
    \item Convergenza. Aumentando il numero di elementi la soluzione numerica deve avvicinarsi a quella esatta. Diminuendo la dimensione degli elementi questi diventano infinitesimi, condizione in cui le deformazioni sono costanti. Dunque per garantire la convergenza le $H $ devono essere in grado di rappresentare uno stato di deformazione costante.\\
    \item Per elementi vicini le funzioni di forma devono dare lo stesso spostamento dei nodi comuni, sia partendo da un elemento che dall'altro. Trovandoci esattamente su un nodo, lo spostamento è pari a quello del nodo stesso, dunque le funzioni di forma devono avere valore unitario ai nodi, così da restituire esattamente lo spostamento nodale.\\
    \item Il grado delle funzioni di forma dipende dal numero di nodi dell'elemento.
\end{enumerate}
