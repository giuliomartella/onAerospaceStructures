\section{Tensori di deformazione in caso di rotazioni e spostamenti finiti}

\subsubsection*{Traslazione rigida}

Consideriamo un punto $P$ e trasliamolo in direzione $x$ di una quantità $a$, di modo che:
\begin{align*}
    &x = X + a \quad\quad &u_x = a\\
    &y = Y &u_y = 0\\
    &z = Z &u_z = 0
\end{align*}

I tensori gradiente di deformazione, Green-Lagrange e piccole deformazioni diventano, rispettivamente: 
\begin{equation*}
\boldsymbol{F} = 
\begin{bmatrix}
\frac{\partial x}{\partial X} & \frac{\partial x}{\partial Y} & \frac{\partial x}{\partial Z} \\
\frac{\partial y}{\partial X} & \frac{\partial y}{\partial Y} & \frac{\partial y}{\partial Z} \\
\frac{\partial z}{\partial X} & \frac{\partial z}{\partial Y} & \frac{\partial z}{\partial Z}
\end{bmatrix} = 
\begin{bmatrix}
   1 & 0 & 0 \\
   0 & 1 & 0 \\
   0 & 0 & 1 \\
\end{bmatrix}
\end{equation*}
Il tensore gradiente di deformazione risente della traslazione rigida.

 \begin{equation*}
\boldsymbol{E} = \frac{1}{2} \left(\boldsymbol{F}^T \boldsymbol{F} - \boldsymbol{I}\right)
= \left(\begin{bmatrix}
   1 & 0 & 0 \\
   0 & 1 & 0 \\
   0 & 0 & 1 \\
\end{bmatrix} 
\begin{bmatrix}
   1 & 0 & 0 \\
   0 & 1 & 0 \\
   0 & 0 & 1 \\
\end{bmatrix} -
\begin{bmatrix}
   1 & 0 & 0 \\
   0 & 1 & 0 \\
   0 & 0 & 1 \\
\end{bmatrix}\right) =
\begin{bmatrix}
   0 & 0 & 0 \\
   0 & 0 & 0 \\
   0 & 0 & 0 \\
\end{bmatrix}
\end{equation*}

\begin{equation*}
\epsilon_{ij} = \frac{1}{2} \left( u_{ji} + u_{ij} \right)
= 
\begin{bmatrix}
\epsilon_{xx} & \epsilon_{xy} & \epsilon_{xz} \\
\epsilon_{yx} & \epsilon_{yy} & \epsilon_{yz} \\
\epsilon_{zx} & \epsilon_{zy} & \epsilon_{zz}
\end{bmatrix}
= 
\frac{1}{2}
\begin{bmatrix}
2 u_{x/x} & u_{x/y} + u_{y/x} & u_{x/z} + u_{z/x} \\
u_{y/x} + u_{x/y} & 2 u_{y/y} & u_{y/z} + u_{z/y} \\
u_{z/x} + u_{x/z} & u_{z/y} + u_{y/z} & 2 u_{z/z}
\end{bmatrix}
=
\begin{bmatrix}
0 & 0 & 0 \\
0 & 0 & 0 \\
0 & 0 & 0
\end{bmatrix}
\end{equation*}



I tensori di Green-Lagrange e piccole deformazioni non risentono della rotazione rigida.

\subsubsection*{Rotazione rigida}


Consideriamo un punto $P$ e ruotiamolo in direzione $y$ di una quantità $\pi/ 2$, di modo che:
\begin{align*}
    &x = Z \quad\quad &u_x = Z-X\\
    &y = Y    &   u_y = 0\\
    &z = -X &u_z = -X-Z
\end{align*}

I tensori gradiente di deformazione, Green-Lagrange e piccole deformazioni diventano, rispettivamente: 
\begin{equation*}
\boldsymbol{F} = 
\begin{bmatrix}
\frac{\partial x}{\partial X} & \frac{\partial x}{\partial Y} & \frac{\partial x}{\partial Z} \\
\frac{\partial y}{\partial X} & \frac{\partial y}{\partial Y} & \frac{\partial y}{\partial Z} \\
\frac{\partial z}{\partial X} & \frac{\partial z}{\partial Y} & \frac{\partial z}{\partial Z}
\end{bmatrix} = 
\begin{bmatrix}
   0 & 0 & -1 \\
   0 & -1 & 0 \\
   -1 & 0 & 0 \\
\end{bmatrix}
\end{equation*}
Il tensore gradiente di deformazione risente della traslazione rigida.




 \begin{equation*}
\boldsymbol{E} = \frac{1}{2} \left(\boldsymbol{F}^T \boldsymbol{F} - \boldsymbol{I}\right)
= \left(\begin{bmatrix}
   0 & 0 & -1 \\
   0 & -1 & 0 \\
   -1 & 0 & 0 \\
\end{bmatrix} 
\begin{bmatrix}
   0 & 0 & -1 \\
   0 & -1 & 0 \\
   -1 & 0 & 0 \\
\end{bmatrix} -
\begin{bmatrix}
   1 & 0 & 0 \\
   0 & 1 & 0 \\
   0 & 0 & 1 \\
\end{bmatrix}\right) =
\begin{bmatrix}
   0 & 0 & 0 \\
   0 & 0 & 0 \\
   0 & 0 & 0 \\
\end{bmatrix}
\end{equation*}

Il tensore di Green-Lagrange non risente delle rotazioni rigide.

\begin{equation*}
\epsilon_{ij} = \frac{1}{2} \left( u_{ji} + u_{ij} \right)
= 
\begin{bmatrix}
\epsilon_{xx} & \epsilon_{xy} & \epsilon_{xz} \\
\epsilon_{yx} & \epsilon_{yy} & \epsilon_{yz} \\
\epsilon_{zx} & \epsilon_{zy} & \epsilon_{zz}
\end{bmatrix}
= 
\frac{1}{2}
\begin{bmatrix}
2 u_{x/x} & u_{x/y} + u_{y/x} & u_{x/z} + u_{z/x} \\
u_{y/x} + u_{x/y} & 2 u_{y/y} & u_{y/z} + u_{z/y} \\
u_{z/x} + u_{x/z} & u_{z/y} + u_{y/z} & 2 u_{z/z}
\end{bmatrix}
=
\begin{bmatrix}
-1 & 0 & 0 \\
0 & 0 & 0 \\
0 & 0 & -1
\end{bmatrix}
\end{equation*}

Il tensore delle piccole deformazioni risente delle rotazioni rigide.





Concludiamo che il tensore di Green-Lagrange si rivela l'unico trasparente alle rototraslazioni rigide.