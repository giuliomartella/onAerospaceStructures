\section{Teorema di Menabrea.}

\begin{enunciatoBox}
   Tra tutte le configurazioni equilibrate, quella congruente rende stazionaria l'energia complementare totale.
\end{enunciatoBox}
 Il teorema scrive condizioni di congruenza, si può quindi usare nel metodo delle forze dopo aver scritto le condizioni di equilibrio.

 \begin{definizioneBox}
     L'energia complementare totale, $\Pi_c$, è la somma dell'energia complementare delle reazioni di vincolo (da vincoli che lavorano su forze infinitesime) e dell'energia complementare di deformazione.\\
     Nel caso di legame sforzo-deformazione conservativo il lavoro complementare di deformazione, $\delta L_d \, = \, \int_{\Omega} \left\{\varepsilon \right\}^T \left\{\delta\sigma \right\} \, dV$, è il differenziale esatto dell'energia complementare di deformazione ($L_d$ o $V_d$).
 \end{definizioneBox}

 Considerando dei vincoli fissi:\\
 Tra tutte le configurazioni equilibrate, quella congruente minimizza l'energia di deformazione.\\

 \begin{definizioneBox}
     Energia di deformazione. Energia che è necessario somministrare al corpo per deformarlo.
     \begin{equation*}
         V_d \, = \, \frac{1}{2}\int_{\Omega} \left\{\varepsilon \right\}^T \left\{\sigma \right\} \, dV
     \end{equation*}
 \end{definizioneBox}

 Questo teorema ha applicazioni nel ricavare l'espressione di $\left\{\sigma \right\}$ nel modello di trave e nello studio delle equazioni dei flussi sorgenti nelle giunzioni (con due o tre file di vincoli e negli incollaggi).

 
 
 
