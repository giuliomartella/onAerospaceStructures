\section{Modello di trave. Definizione e approccio con il metodo degli spostamenti.}

Si definisce trave il volume generato dalla traslazione di una figura piana(sezione) lungo un asse perpendicolare ad essa. Per convenzione $z$ è l'asse della trave e $(x,y)$ rappresenta il piano della sezione. Rimane l'ipotesi di snellezza.\\

Per studiare l'approccio del metodo degli spostamenti nella risoluzione delle travi vanno aggiunte ulteriori ipotesi esplicative.
\begin{enumerate}
    \item Gli spostamenti, di un qualunque punto $P$ della sezione, devono essere esprimibili in funzione degli spostamenti dell'intersezione $Q$ dell'asse della trave con la sezione a cui $P$ appartiene. Dunque gli spostamenti di $P$ dipendono solo dagli spostamenti di $Q$.
    \begin{align*}
    \{s_P\} &= \begin{Bmatrix} s_{Px} \\ s_{Py} \\ s_{Pz} \end{Bmatrix}, \quad
    \{w_Q\} = \begin{Bmatrix} w_{Qx} \\ w_{Qy} \\ w_{Qz}  \end{Bmatrix}
    \quad \rightarrow \quad
    \{s_P\} = f\big( \{w_Q\} \big)
\end{align*}
    \item Se il punto $Q$ si sposta di $w_{z}$, tutti i punti della sezione si spostano di $w_{z}$.
    \item La sezione della trave rimane sempre perpendicolare all'asse. Se la trave flette, soggetta a un carico diretto come $y$, la sua sezione ruoterà intorno all'asse $x$. Affinché la sezione rimanga perpendicolare all'asse ogni suo punto sarà soggetto a uno spostamento $s_z$, tale che:
    \begin{align*}
        s_z &= -y\, sin\theta \approx -y \,\theta    \quad \quad \quad \quad \theta = \frac{\partial w_y}{\partial z} = w_y'\\
        &= -w_y'  y
    \end{align*}
    Quindi per un generico punto $P$ della sezione vale
    \begin{equation*}
        s_z \,=\,  w_z \,-\,w'_x  x\, -\,w'_y  y
    \end{equation*}


\end{enumerate}



