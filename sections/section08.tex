\section{Principio dei lavori virtuali e Principio dei lavori virtuali complementari}


\begin{definizioneBox}
Spostamento virtuale. Sistema di spostamenti arbitrario, infinitesimo e congruente. Sono
\begin{compactitem}
    \item arbitrari: comportano la molteplicità dei lavori virtuali,\\
    \item infinitesimi: non allontanano dalla configurazione di equilibrio in esame,\\
    \item congruenti: rispettano i vincoli di congruenza interna ed esterna.
\end{compactitem}
\end{definizioneBox}
\begin{definizioneBox}
Lavoro virtuale. Lavoro compiuto dal sistema delle forze reali per un campo di spostamenti virtuali.
\end{definizioneBox}

\subsubsection*{Principio dei lavori virtuali}
\begin{enunciatoBox}
    Condizione necessaria e sufficiente per l'equilibrio di una struttura è che tutti i possibili lavori virtuali siano minori o uguali a zero.
\end{enunciatoBox}



\begin{compactitem}
    \item Questo consente di scrivere condizioni/equazioni di equilibrio.\\
    \item Dalla definizioni capiamo che possono esserci più lavori virtuali.\\
    \item Forze reali e spostamenti reali sono la storia del corpo e sono dovuti ai carichi.\\
    \item Il lavoro virtuale non è l'energia spesa per spostare il corpo.
\end{compactitem}

Il Principio dei Lavori virtuali vale sempre.

\begin{esempioBox}
Preso un provino in condizione deformata:
\begin{compactitem}
    \item Il lavoro di deformazione è servito per arrivare all'equilibrio della nuova condizione deformata.
    \item Il lavoro virtuale è causato dalla forza reale $\tilde{F}$ rispetto a uno spostamento virtuale infinitesimo $\delta \tilde{s}$.
\end{compactitem}
I due lavori sono diversi.

\end{esempioBox}



\subsubsection*{Principio dei Lavori Virtuali Complementari}
\begin{definizioneBox}
    Lavoro virtuale complementare. Lavoro compiuto dal sistema di spostamenti reali per un sistema di forze virtuali.
\end{definizioneBox}
\begin{definizioneBox}
    Sistema di forze virtuali. Sistema di forze arbitrario, infinitesimo ed equilibrato.\\
    Infinitesimo perché è un sistema di forze piccole, tali da non deformare la struttura. Permette di studiare la configurazione indeformata, nell'ambito degli spostamenti infinitesimi.
\end{definizioneBox}

\begin{enunciatoBox}
    Condizione necessaria e sufficiente per la congruenza di una struttura è che tutti i possibili lavori virtuali complementari siano uguali a zero.

\end{enunciatoBox}

\begin{compactitem}
    \item Scrive condizioni/equazioni di congruenza.
    \item Vale solo per spostamenti infinitesimi.
\end{compactitem}

\begin{esempioBox}
    Consideriamo una trave incastrata orizzontale di lunghezza $l$ alla cui estremità è applicata una forza verticale $F$.
    Nella condizione indeformata il momento all'incastro vale esattamente $ M = Fl$.
    Nella condizione deformata la trave si flette, il braccio della forza diventa $l_1<l$ e il momento all'incastro risulta $M_1 = Fl_1<M$.
    Nel primo caso forze e spostamenti sono infinitesimi, nel secondo sono finiti.
\end{esempioBox}








