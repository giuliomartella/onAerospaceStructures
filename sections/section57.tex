\section{Ingobbamento. Dimostrazione}

L'ingobbamento è il movimento dei punti della sezione in direzione $z$ provocato dalle $\tau(q)$ dei pannelli.

Sono sotto esame gli spostamenti lungo $z$ per effetto delle $\tau$. Per calcolare $s_z$ uso  il PLVC su un pannello generico.\\
Nel sistema reale considero gli spostamenti di due estremi sullo stesso lato, in direzione $z$: $s_1$ ed $s_2$, le rotazioni attorno all'asse a distanza $dz$: $\theta$ e $\theta+d\theta$, i flussi attorno al pannello $q$.\\
Nel sistema fittizio considero i flussi unitari che generano un momento attorno $z$ uguale a $\delta M=2\Omega$. 

\begin{align*}
    \delta L_e\,&=\, 2\Omega(\theta+d\theta)-2\Omega\theta -s_1dz+s_2dz\\
    &=\, 2\Omega\theta + (s_2-s_1)dz\\
    \delta L_d \,&=\, \frac{q\delta q\,l}{Gt}dz\,=\, \frac{q\,l}{Gt}dz
\end{align*}

Per cui, eguagliando i lavori di deformazione $ \delta L_e\, =\,\delta L_d$:
\begin{align*}
    2\Omega\theta + (s_2-s_1)dz \,&=\,\frac{q\,l}{Gt}dz\\
    s_2-s_1 \,&=\,\frac{q\,l}{Gt}- 2\Omega\dot{\theta}\\
     s_{2j}-s_{1j} \,&=\,\frac{q_j\,l_j}{Gt_j}- 2\Omega_{Oj}\dot{\theta}\\
\end{align*}

Abbiamo trovato quindi la formula per l'ingobbamento, questa non consente di calcolare la traslazione assoluta ma solo quella relativa, dunque la traslazione calcolata è rigida.\\
Inoltre $\Omega_{Oj}$ dipende dall'asse considerato, per lo stesso pannello cambiando il polo di riferimento cambia anche l'ingobbamento. L'ingobbamento è quindi affetto anche da rotazione calcolata rigida.\\
Nonostante questo permette di capire qualitativamente come si deformano le sezioni.