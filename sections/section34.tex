\section{Passaggio dalla sezione in parete sottile alla sezione a guscio.}

Da questo punto considereremo strutture composte da pannelli sottili, correnti esili e diaframmi (ordinate o centine). Prendiamo ad esempio una struttura formata da quattro pannelli collegati da quattro correnti, estraiamo il pannello per analizzarlo. Per questo continua a valere il modello di trave, per cui le $\sigma$ sono note, le $\tau$ rispettano l'equazione dei flussi sorgenti in seconda forma. Adesso consideriamo:

\begin{enumerate}
    \item Le $\tau$ sono tangenti al contorno del pannello.\\
    \item Non consideriamo l'andamento delle $\tau$ a farfalla perché la rigidezza torsionale che generano è nulla, le $\tau$ sono quindi dirette sempre nello stesso verso. Inoltre consideriamo due linee di controllo che includono bordi opposti, per cui vale $ \Phi\, = -\, \frac{T_y}{J_x}S'_x $, dove $S'_x = A'y_{CA}$. Dato che $y_{CA}$ può variare al massimo dello spessore della trave $t<<L$, i flussi di taglio attraverso le due linee di controllo saranno molto simili. Assumiamo quindi che le $\tau$ non varino nello spessore del pannello. Ipotizziamo anche che i pannelli abbiano spessore costante.\\
    \item Le $\tau$ sono parallele e simmetriche rispetto la linea media $l.m.$.
    \item A questo punto:
    \begin{definizioneBox}
        Flusso di taglio nel pannello:
        \begin{equation*}
            q(l) = \int_{-\frac{t}{2}}^{+\frac{t}{2}} \, \tau \,dt
        \end{equation*}
        Dato che le $\tau $ sono costanti lungo lo spessore rimane:
        \begin{equation*}
            q(l) \,= \, \tau_{media} \,t
        \end{equation*}
        Il flusso di taglio $q(l)$ è un vettore diretto come la linea media, è la risultante delle $\tau$ nello spessore. 
    \end{definizioneBox}
    Dopo aver svolto l'integrale sopra lo spessore del pannello diventa irrilevante, posso quindi descrivere il pannello attraverso la sua linea media.
    In teoria $\tau$ può cambiare lungo il pannello, questo comporta:
    \begin{align*}
        &\Phi\, = \, q_{ju} -q_{je}\,=\, -\frac{T}{J}S'\\
        &q_{ju} \,\neq\, q_{je}  \quad \quad \rightarrow  \quad \quad  -\frac{T}{J}S'\,\neq\, 0 \quad \quad \rightarrow  \quad \quad S'\,\neq\, 0
    \end{align*}\\
    \item Definiamo la risultante delle $\tau$ nello spessore come:
    \begin{equation*}
        p(l) = \int_{-\frac{t}{2}}^{+\frac{t}{2}} \, \sigma \,dt\,= \, \sigma_{media} \,t
    \end{equation*}
    Questo perché le $\sigma$ variano poco nello spessore essendo $\sigma = \frac{M}{J}y$ e $t<<L$.
    $p$ e $q$ sono ortogonali.\\
    \item I correnti sono esili, ovvero piccoli rispetto alle dimensioni della sezione, e lontani rispetto agli assi principali d'inerzia. Valutiamo  il loro contributo al momento d'inerzia $J_{xi}= A_iy_i^2 + J_{\bar{x}}$ , dalle ipotesi sopra $A_iy_i^2 >> J_{\bar{x}}$, ovvero il momento di trasporto è molto maggiore del momento di trasporto locale. Se due correnti hanno la stessa area e sono posizionati alla stessa distanza dall'origine del sistema, allora danno lo stesso contributo al momento d'inerzia totale, perde di significato la forma del corrente.\\
    \item Ignorando la forma del corrente non possiamo più distinguere le $\sigma$ tra un punto e un altro della sezione di questo. Segue che le $\sigma$ sono costanti nei correnti.\\
    \item Definisco l'azione assiale nel corrente i-esimo come:
    \begin{equation*}
        N_i \,=\, \sigma_i A_i
        \quad  \quad  \quad  \quad
        \sigma_i\,=\,
         \frac{T_z}{ \bar{A}} 
        + \frac{M_x}{ J_x} y_i
        - \frac{M_y}{ J_y} x_i
    \end{equation*}
    Modellando i pannelli solo attraverso la linea media i correnti diventano l'unico elemento dotato di area nella sezione.   \\
    \item Dipendono dall'area $T_x = \int_{\bar{A}}\tau_x\,dA$ e  $T_y = \int_{\bar{A}}\tau_y\,dA$. Sapendo che l'area dei correnti è piccola rispetto a quella della sezione (dato che i pannelli sono sviluppati in lunghezza), il contributo delle $\tau$ nei correnti è trascurabile rispetto a quello nei pannelli, lo supponiamo nullo.
    
    
    
\end{enumerate}






