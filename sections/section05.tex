\section{Punto di vista Lagrangiano ed Euleriano}

Considerando un corpo che si deforma e si muove nello spazio. Siano $P$ un punto, $\mathbf{X}$ la sua posizione nella condizione indeformata, $\mathbf{u}$ la deformazione, e quindi la mappa dalla condizione indeformata a quella deformata e sia $\mathbf{x} = \mathbf{X} +\mathbf{u}$ la posizione di $P$ nella condizione deformata. Esistono due approcci per descrivere il movimento.

\begin{itemize}
    \item \textbf{Punto di vista materiale o Lagrangiano}: Si fa riferimento alla configurazione iniziale e tutte le variabili sono rappresentate in funzione di $\mathbf{X}$.
    
    \item \textbf{Punto di vista spaziale o Euleriano}: Tutte le variabili sono espresse in funzione di $\mathbf{x}$ e la configurazione di riferimento è quella finale.
\end{itemize}

Con il punto di vista materiale posso identificare il punto prima di applicare i carichi, posso quindi avere la storia di tale punto seguendolo mentre vengono applicati i carichi.\\
Nella zona di strizione di un provino i punti si allontanano molto fra di loro, segue che una griglia regolare applicata al provino prima di applicare i carichi diventa irregolare (e localmente più o meno fitta), di conseguenza nella strizione si perde regolarità e dettaglio.
Diventa opportuno applicare una griglia euleriana che invece mantiene la sua spaziatura.


