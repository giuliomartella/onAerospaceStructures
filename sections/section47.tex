\section{Definizione e ruolo dei diaframmi.}


\begin{definizioneBox}
    I diaframmi sono componenti strutturali trasversali della trave a semiguscio. Assumono il nome di ordinate se fanno parte della fusoliera, di centine se fanno parte dell'ala.
    
    La prima funzione è quella di mantenere la forma della sezione, permettendo che la trave risponda al momento torcente ruotando rigidamente. La seconda funzione è di introdurre i carichi, distribuiti o concentrati (in questo caso con ordinate di forza), nella struttura. La terza funzione consiste nel limitare i fenomeni di instabilità di correnti e pannelli, agendo come rompitratta riducono la lunghezza di libera inflessione.
\end{definizioneBox}

Di solito sono ottenibili lavorando delle lamiere o fresando il metallo su un piano. Gli aerei di grandi dimensioni posso averli sotto forma di travature reticolari.\\
Il diagramma deve mantenere la forma della sezione, per questo sul piano $(x,y)$ è infinitamente più rigido della sezione considerata con solo correnti e pannelli. Ne consegue che il movimento della sezione a semiguscio nel piano $(x,y)$ è un movimento rigido, ovvero tutti i punti ruotano della stessa quantità.
La torsione è una rotazione rigida di tutta la sezione attorno all'asse $z$, uscente dal centro di taglio.\\


Per introdurre il carico concentrato nella struttura si potrebbe usare una piastra imbullonata al pannello, con il rischio di strapparlo perché troppo sottile. Per ovviare al problema si usa una centina di forza, una struttura collegata su tutto il contorno ai pannelli, come una sezione piena. Il carico indotto fa lavorare la centina nel suo piano (dove è rigida) e distribuisce la forza a tutti i pannelli con cui è collegata, in questo modo la forza è distribuita in maniera uniforme nella struttura.\\


La centina si può studiare come una travatura reticolare in cui i flussi $q_j$, provenienti dai pannelli, si trasformano in azioni assiali distribuite applicate al contorno (costituito da travi).\\
Nel caso in cui, ad esempio, la centina sia formata da un pannello piegato, forato e imbutito. Il flusso si scarica solo sulle alette del contorno. La centina qui si può modellare come una trave a sezione non costante (in cui la corda è l'asse della trave). I flussi trasmessi dalle giunzioni si trasformano in momenti flettenti distribuiti e azioni assiali distribuite.






