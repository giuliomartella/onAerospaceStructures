\section{Schematizzazione delle giunzioni}

\begin{definizioneBox}
    Giunzioni, elementi che trasferiscono i carichi all'interno della struttura composta da correnti, pannelli e diaframmi.
\end{definizioneBox}

Negli schemi disegniamo i flussi di giunzione nel piano, ma in realtà scorrono lungo l'asse $z$. Il flusso è trasmesso lungo i pannelli attraverso la linea di giunzione, infatti la forza trasmessa è longitudinale, diretta come $z$.

Posso avere giunzioni semplici, a doppia fila, con sovrapposizione.

\begin{esempioBox}
    Consideriamo delle giunzioni, dall'alto verso il basso incontriamo due pannelli e poi un corrente, il corrente e poi collegato a un terzo pannello da destra. La prima giunzione collega i primi due pannelli, la seconda arriva al corrente, la terza arriva al terzo pannello.
    Otteniamo i flussi usando l'equazione dei flussi sorgenti:
    \begin{align*}
        q_{G_1} \,&=\,q_1\\
        q_{G_2} +q_2-q_1\,&=\,0   \qquad \qquad     q_{G_2} \,=\,q_1-q_2\\
        q_{G_3} \,&=\,q_3\\
    \end{align*}
\end{esempioBox}

Possiamo ottenere la giunzione ad esempio per mezzo di una chiodatura. Si definisce passo $p$ la distanza fra due chiodi successivi.
Le forze complessive trasmesse sono $qL$ mentre la forza su ogni chiodo è:
\begin{equation*}
    \frac{F_{tot}}{n_{chiodi}} \,=\, \frac{qL}{n} \,=\,qp
\end{equation*}

Esistono due tipologie di giunzioni:
\begin{compactitem}
    \item Continue (saldature, incollaggi): il trasferimento dei carichi da un pannello all'altro, o ad un corrente, avviene in modo distribuito e continuo.\\
    \item Discontinue (chiodature, vite-dado, saldature per punti): il trasferimento dei carichi avviene in modo puntuale e discreto.
\end{compactitem}
