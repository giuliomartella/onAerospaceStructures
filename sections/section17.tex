\section{Modello di trave. Stato di sforzo e di deformazione.}

Nel modello di trave, ponendo $z$ come direzione dell'asse e $(x,y)$ come piano della sezione, le uniche componenti non nulle di $\left\{\sigma\right\}$ sono:
\begin{equation*}
    \sigma_{zz}\, =\, \sigma, \quad\quad\quad  \sigma_{zx}\, =\, \tau_x, \quad\quad\quad \sigma_{zy}\, =\, \tau_y 
\end{equation*}
Per simmetria del tensore degli sforzi sono diverse da zero, nel tensore $\boldsymbol{\sigma}$, anche le componenti $\sigma_{xz}$ e $\sigma_{yz}$.\\
Adesso eseguo dei tagli alla trave
\begin{enumerate}
    \item Un taglio con piano di normale $x$,\\
    \item Un taglio con piano di normale $y$,\\
    \item Un taglio con piano di normale generica, combinazione lineare di $x$ e $y$.
\end{enumerate}
Sempre per simmetria del tensore degli sforzi ognuno di questi tagli evidenzia una $\tau$ sul piano del taglio in direzione $z$ e una corrispondente sul piano della sezione, in direzione della normale del piano di taglio. Questi tagli idealmente possono arrivare fino alle superfici laterali, che però sono scariche per ipotesi.\\
Valutiamo, attraverso la relazione di Cauchy $\boldsymbol{\sigma}^T\mathbf{n}=\mathbf{t_n}$, il vettore degli sforzi sulla superficie laterale con normale $x$.
 
\begin{equation*}
\begin{bmatrix}
\sigma_{xx} & \sigma_{yx} & \sigma_{zx} \\
\sigma_{xy} & \sigma_{yy} & \sigma_{zy} \\
\sigma_{xz} & \sigma_{yz} & \sigma_{zz}
\end{bmatrix}
\begin{bmatrix}
1 \\
0 \\
0
\end{bmatrix}
=
\begin{bmatrix}
\sigma_{xx} \\
\sigma_{xy} \\
\sigma_{xz}
\end{bmatrix}
= \mathbf{0}   
\quad\quad\quad \rightarrow \quad\quad\quad 
\begin{matrix}
\sigma_{xx} = 0 \\
\sigma_{xy} = 0 \\
\sigma_{xz} = \sigma_{zx}  = \tau_x = 0
\end{matrix}
\end{equation*}

Mentre i primi due risultati sono prescritti dallo sforzo nel modello di trave, il terzo viene imposto sul contorno della sezione. Analogamente possiamo ottenere che sulla faccia esterna con normale diretta come $y$: $\tau_y = 0$. Questa analisi prova che le $\tau$ siano tangenti al contorno della sezione e, di conseguenza, che tutte le $\tau$ normali al bordo della sezione risultino nulle.\\

Per valutare le deformazioni partiamo dalla notazione di Voigt per un materiale elastico, lineare e isotropo.
\begin{equation*}
\left\{
\begin{array}{c}
\varepsilon_{xx} \\
\varepsilon_{yy} \\
\varepsilon_{zz} \\
\gamma_{yz} \\
\gamma_{xz} \\
\gamma_{xy}
\end{array}
\right\}
=
\left[
\begin{array}{cccccc}
\displaystyle \frac{1}{E} & \displaystyle -\frac{\nu}{E} & \displaystyle -\frac{\nu}{E} & 0 & 0 & 0 \\
\displaystyle -\frac{\nu}{E} & \displaystyle \frac{1}{E} & \displaystyle -\frac{\nu}{E} & 0 & 0 & 0 \\
\displaystyle -\frac{\nu}{E} & \displaystyle -\frac{\nu}{E} & \displaystyle \frac{1}{E} & 0 & 0 & 0 \\
0 & 0 & 0 & \displaystyle \frac{1}{G} & 0 & 0 \\
0 & 0 & 0 & 0 & \displaystyle \frac{1}{G} & 0 \\
0 & 0 & 0 & 0 & 0 & \displaystyle \frac{1}{G}
\end{array}
\right]
\left\{
\begin{array}{c}
\sigma_{xx} \\
\sigma_{yy} \\
\sigma_{zz} \\
\sigma_{yz} \\
\sigma_{xz} \\
\sigma_{xy}
\end{array}
\right\}
\end{equation*}

Considerando solo le componenti di sforzo non nulle nel modello di trave:
\begin{equation*}
    \begin{bmatrix}
\varepsilon \\
\gamma_{x} \\
\gamma_{y}
\end{bmatrix}
=
\begin{bmatrix}
\displaystyle \frac{1}{E} & 0 & 0 \\
0 & \displaystyle \frac{1}{G} & 0 \\
0 & 0 & \displaystyle \frac{1}{G}
\end{bmatrix}
\begin{bmatrix}
\sigma_x \\
\tau_{x} \\
\tau_{y}
\end{bmatrix}
\end{equation*}

Sono presenti anche le componenti:
\begin{equation*}
    \varepsilon_{xx} = -\frac{\nu}{E}\sigma \quad\quad\quad\quad  \varepsilon_{yy} = -\frac{\nu}{E}\sigma
\end{equation*}
Queste deformazioni, dovute a vincoli interni o condizioni di simmetria, non compiono lavoro e non sono quindi considerate tra le incognite del problema.\\

Riguardo l'andamento di sforzi e deformazioni nella trave ritroviamo tre componenti:
 \begin{equation*}
        \sigma = \frac{T_z}{\bar{A}} +  \frac{M_x}{J_x}y -\frac{M_y}{J_y}x \quad\quad\quad \varepsilon = \frac{T_z}{E\bar{A}} +  \frac{M_x}{EJ_x}y -\frac{M_y}{EJ_y}x 
 \end{equation*}
 Il primo contributo è uniforme e costante lungo tutta la sezione. Il secondo ha una variazione lineare lungo $y$ mentre il terzo ha una variazione lineare lungo $x$. Dividendo per il modulo di Young si ottiene l'analoga distribuzione della deformazione assiale.



