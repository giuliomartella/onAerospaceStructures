\section{Tensori coniugati. Dimostrazione mediante PLV}

\begin{definizioneBox}
Tensori coniugati. Due tensori, uno di sforzo e uno di deformazione, sono detti energeticamente coniugati
se il loro prodotto scalare doppio (ovvero la contrazione su due indici) rappresenta il lavoro di
deformazione (o lavoro virtuale interno) a cui partecipa il materiale, integrato sul volume
dell'elemento considerato.
\end{definizioneBox}
Esprimo  il lavoro di deformazione:
\begin{align*}
    \delta L_d \, &=\,  \int_{\Omega} \delta u_{j/i}  \sigma_{ij}  \, dV  
      \quad\quad\quad \left( \boldsymbol{\sigma} = \boldsymbol{\sigma}^T \right)\\
    &= \,  \int_{\Omega}  \left( \frac{1}{2} \delta u_{i/j}  \sigma_{ij} \, + \,        \frac{1}{2} \delta u_{j/i}  \sigma_{ij} \right) \, dV  \\
    &=  \,  \int_{\Omega}  \frac{1}{2} \left( \delta u_{i/j}  \, + \,       \delta u_{j/i}  \right)\,\sigma_{ij}  \, dV  \\
     \delta L_d \, &= \,  \int_{\Omega}  \delta \varepsilon_{ij}  \,\sigma_{ij}  \, dV 
\end{align*}

Il tensore coniugato al tensore degli sforzi $\boldsymbol{\sigma}$ è il tensore delle piccole deformazioni $\boldsymbol{\varepsilon}$.\\

Quando si analizzano deformazioni finite si lavora di solito nella configurazione di riferimento (Lagrangiana). In questa configurazione cerchiamo il tensore coniugato al tensore di Green-Lagrange, ovvero alla misura Lagrangiana della deformazione finita. 

\begin{definizioneBox}
    Il II Tensore di Piola-Kirchhoff, ovvero il tensore coniugato al tensore di Green-Lagrange per deformazioni finite.
    \begin{equation*}
        \boldsymbol{\Sigma}_{\mathrm{II}} \,= \, \det (\boldsymbol{F} )\, \boldsymbol{F}^{-1} \boldsymbol{\sigma} \boldsymbol{F}^T
    \end{equation*}
\end{definizioneBox}

Possiamo quindi riscrivere l'espressione del PLV come segue.
\begin{align*}
     & \int_{S_{\sigma}}  \delta u_j T_j \, dA  \, +\,\int_{\Omega} \delta u_{j}  b_j  \, dV \, = \, \int_{\Omega} \delta \varepsilon_{ij}  \sigma_{ij}  \, dV   \\
    &  \int_{S_{\sigma}} \delta \mathbf{u} \cdot \mathbf{T} \, dA 
    \,+\, \int_{\Omega} \delta \mathbf{u} \cdot \mathbf{b} \, dV 
    \,=\, \int_{\Omega} \delta \boldsymbol{\varepsilon} : \boldsymbol{\sigma} \, dV
\end{align*}








