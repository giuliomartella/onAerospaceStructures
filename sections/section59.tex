\section{Metodo degli spostamenti per il calcolo dei flussi in una sezione a semiguscio.}

Considero un generico pannello. I carichi sono imposti dai flussi uscenti dagli estremi: $q_{1j}$ e $q_{2j}$, sul pannello scorre $q_j$, consegue $q_{1j} = -q_j$ e $q_{2j}=q_j$.
\begin{align*}
    s_{2j}-s_{1j} \,&=\,\frac{q_j\,l_j}{Gt_j}- 2\Omega_{Oj}\dot{\theta}\\
    q_j \,&=\,\frac{Gt_j}{l_j} (s_{2j}-s_{1j}- 2\Omega_{Oj}\dot{\theta})\\
\end{align*}
\begin{equation*}
    \begin{cases}
        q_{1j} \,&=\,-\frac{Gt_j}{l_j} (s_{2j}-s_{1j}- 2\Omega_{Oj}\dot{\theta})\\
        q_{2j} \,&=\,\frac{Gt_j}{l_j} (s_{2j}-s_{1j}- 2\Omega_{Oj}\dot{\theta})\\
        M_{Oj} \,&=\, 2\Omega_{Oj}q_j \,=\,   2\Omega_{Oj}\left(      \frac{Gt_j}{l_j} (s_{2j}-s_{1j}- 2\Omega_{Oj}\dot{\theta})\right)\\
    \end{cases}
\end{equation*}


Definisco allora:
 \begin{equation*}
     \left\{Q\right\}_j = \left\{
    \begin{array}{c}
    q_{1j} \\ q_{2j} \\ M_{Oj}
    \end{array}
    \right\} \gamma
   \quad\quad\quad\quad
    \left\{u\right\} = \left\{
    \begin{array}{c}
    s_{1j} \\ s_{2j} \\ \dot{\theta}
    \end{array}
    \right\} 
 \end{equation*}

Per cui vale: 
\begin{equation*}
    \left\{Q\right\}_j = [K]_j  \left\{ u\right\}_j
\end{equation*}



\begin{equation*}
\left\{
\begin{array}{c}
q_{1j}\\
q_{2j} \\
M_{Oj}
\end{array}
\right\}
=
\frac{Gt_j}{l_j}
\begin{bmatrix}
1 & -1 & -2\Omega_{Oj} \\
-1 & 1 & 2\Omega_{Oj} \\
-2\Omega_{Oj} & 2\Omega_{Oj} & 4\Omega_{Oj}^2
\end{bmatrix}
\left\{
\begin{array}{c}
s_{1j} \\
s_{2j} \\
\dot{\theta}
\end{array}
\right\}
\end{equation*}

Ritrovandoci nel metodo degli spostamenti usiamo il PLV
\begin{align*}
    \delta L_d  &= \sum_{j=1}^m \delta L_{di} \\
   \delta L_{dj}   &=\left\{\delta u\right\}_j^T   \left\{ Q\right\}_j\\
        &=\left\{\delta u\right\}_j^T  [K]_j  \left\{u\right\}_j
\end{align*}
Il vettore degli spostamenti per ogni pannelli si ricava dal vettore degli spostamenti di tutti i punti della sezione attraverso la matrice di incidenza $[A]_j$, di dimensioni $3\times(m+1)$:
\begin{equation*}
    \left\{ u\right\}_j = [A]_j  \left\{U\right\}
\end{equation*}

\begin{esempioBox}
    Calcoliamo la matrice di rigidezza per una sezione rettangolare a due celle, con 6 correnti e 7 pannelli. In particolare la matrice ha dimensioni $3\times(m+1)=3\times7$. \\
    Consideriamo il pannello 3, che ha come estremi il corrente 3 e il corrente 4. In due dimensioni inoltre $\dot{\theta}_3=\dot{\theta}$, l'ultima riga è uguale per qualsiasi pannello.

    \begin{equation*}
[A]_3 = \begin{bmatrix}
0 & 0 & 1 & 0 & 0 & 0 & 0 \\
0 & 0 & 0 & 1 & 0 & 0 & 0 \\
0 & 0 & 0 & 0 & 0 & 0 & 1
\end{bmatrix}
\end{equation*}
Il pannello 7 per esempio ha come estremi i correnti 2 e 5, per cui:
 \begin{equation*}
[A]_7 = \begin{bmatrix}
0 & 0 & 0 & 0 & 1 & 0 & 0 \\
0 & 1 & 0 & 0 & 0 & 0 & 0 \\
0 & 0 & 0 & 0 & 0 & 0 & 1
\end{bmatrix}
\end{equation*}
    
\end{esempioBox}

Consideriamo quindi il lavoro virtuale per l'intera struttura:
\begin{align*}
 \delta L_{dj}  &=\left\{\delta u\right\}_j^T  [K]_j  \left\{u\right\}_j\\
 &=\left\{\delta U\right\}^T [A]_j^T [K]_j  [A]_j\left\{U\right\}\\
    \delta L_d  &= \sum_{j=1}^m \delta L_{di} \\
        &= \sum_{j=1}^m\left\{\delta U\right\}^T [A]_j^T [K]_j  [A]_j\left\{U\right\} \\
        &=\left\{\delta U\right\}^T   \left( \sum_{j=1}^m    [A]_j^T [K]_j  [A]_j    \right)\left\{U\right\}\\
        &=\left\{\delta U\right\}^T    [K]\left\{U\right\}
\end{align*}

Per scrivere il termine noto considero:
\begin{align*}
 \delta L_{dj}  &=\left\{\delta u\right\}_j^T   \left\{Q\right\}_j\\
 &=\left\{\delta U\right\}_j^T [A]_j^T\left\{Q\right\}_j\\
    \delta L_d  &= \sum_{j=1}^m  \left\{\delta U\right\}_j^T [A]_j^T\left\{Q\right\}_j \\
        &=  \left\{\delta U\right\}_j^T \sum_{j=1}^m [A]_j^T\left\{Q\right\}_j
\end{align*}

Notiamo che i primi $n$ termini della sommatoria (per righe) sono i flussi entranti nei correnti. Riga per riga questa sommatoria coincide con il flusso totale entrante nel corrente (con un meno davanti):
\begin{equation*}
    \Phi_i=-\frac{T_x}{J_y}S'_{yi} - \frac{T_y}{J_x}S'_{xi}
\end{equation*}

\begin{equation*}
    \sum_{j=1}^m [A]_j^T\left\{Q\right\}_j =
    \frac{T_x}{J_y}\left\{
    \begin{array}{c}
    S'_{y1} \\
    S'_{y2} \\
    \vdots \\
    S'_{yn}\\0
    \end{array}
    \right\}
    + \frac{T_y}{J_x}\left\{
    \begin{array}{c}
    S'_{x1} \\
    S'_{x2} \\
    \vdots \\
    S'_{xn}\\0
    \end{array}
    \right\}
    +\left\{
    \begin{array}{c}
    0 \\
    0 \\
    \vdots \\0\\
    M_O(T_x,T_y,M_z)
    \end{array}
    \right\}
\end{equation*}

Notiamo che in questa equazione tutti i termini sono noti e posso scrivere:
\begin{equation*}
    \left\{\delta U\right\}^T    [K]\left\{U\right\}= \left\{\delta U\right\}^T   \left(   
    \frac{T_x}{J_y}\left\{
    \begin{array}{c}
    S'_{y1} \\
    S'_{y2} \\
    \vdots \\
    S'_{yn}\\0
    \end{array}
    \right\}
    + \frac{T_y}{J_x}\left\{
    \begin{array}{c}
    S'_{x1} \\
    S'_{x2} \\
    \vdots \\
    S'_{xn}\\0
    \end{array}
    \right\}
    +\left\{
    \begin{array}{c}
    0 \\
    0 \\
    \vdots \\0\\
    M_O(T_x,T_y,M_z)
    \end{array}
    \right\}
    \right)
\end{equation*}

Per cui ottengo il sistema lineare:
\begin{align*}
     [K]\left\{U\right\}&= \left\{Q\right\}\\
     &=  \left(   
    \frac{T_x}{J_y}\left\{
    \begin{array}{c}
    S'_{y1} \\
    S'_{y2} \\
    \vdots \\
    S'_{yn}\\0
    \end{array}
    \right\}
    + \frac{T_y}{J_x}\left\{
    \begin{array}{c}
    S'_{x1} \\
    S'_{x2} \\
    \vdots \\
    S'_{xn}\\0
    \end{array}
    \right\}
    +\left\{
    \begin{array}{c}
    0 \\
    0 \\
    \vdots \\0\\
    M_O(T_x,T_y,M_z)
    \end{array}
    \right\}
    \right)     
\end{align*}

Potendo scrivere solo $(n-1) $ equazioni dei flussi sorgenti, scritta in questo modo $K$ è labile. \\
Eliminare una riga e colonna equivale a fissare un riferimento di potenziale (ad esempio fissare arbitrariamente uno spostamento di torsione o un flusso di riferimento), evitando la non unicità della soluzione dovuta al fatto che i flussi sono definiti solo a meno di una costante additiva.\\
Elimino una riga ed una colonna da $[K]$, purché non siano le ultime, che contengono il momento.
\begin{align*}
    [K]\left\{U\right\}&= \left\{Q\right\}\\
    \left\{U\right\}&= [K]^{-1}\left\{Q\right\} \qquad\qquad
     \left\{u\right\}_j= [A]_j\left\{U\right\}\\
      \left\{Q\right\}_j&= [K]_j\left\{u\right\}_j\\
\end{align*}



   