\section{Calcolo del fattore di taglio per una sezione a semiguscio.}

I fattori di taglio per una sezione a semiguscio, derivanti dalla formulazione per l'energia di deformazione per le $\tau$:
\begin{equation*}
    V_{d\tau}= \frac{1}{2} \int_l 
    \left[
        \frac{T_x^2}{G A^*_x} 
        + \frac{T_y^2}{G A^*_y} 
        + \frac{M_z^2}{G J_t}
    \right] dz
\end{equation*}

sono $ \frac{1}{G A^*_x} $ e $\frac{1}{G A^*_y} $. Consideriamo una semplice sezione rettangolare di dimensioni $a\times h$.
\begin{enumerate}
    \item Applico un $T_y$ puro, ovvero passante per il CT, per simmetria per questa sezione cade sul baricentro.\\
    \item Sfrutto la simmetria della geometria e dei carichi per calcolare i flussi di taglio: $q_j=\frac{T_y}{2h}$ sui pannelli verticali e zero su quelli orizzontali.\\
    \item Eguaglio le due espressioni per l'energia di deformazione per unità di lunghezza:
        \begin{equation*}
            V_d   =\, \frac{1}{2G} \,\sum^m_{j=1}\frac{q_j^2l_j}{t_j}
            \quad\quad\quad
             V_{d}= \frac{1}{2} \frac{T_y^2}{G A^*_y} 
        \end{equation*}\\
\end{enumerate}

Nel nostro esempio risulta:
\begin{align*}
    \frac{1}{2Gt} \left( \frac{2T_y^2}{4h^2}h \right)\,=\,\frac{1}{2} \frac{T_y^2}{G A^*_y} \\
     \frac{1}{G A^*_y}\,=\,  \frac{1}{2G th}
\end{align*}