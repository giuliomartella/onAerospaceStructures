\section{Teoria della laminazione}

La teoria si basa su un adattamento del modello di piastra, si ha una serie di lamine di materiali diverso e/o con orientazioni diverse.
Introduciamo l'ipotesi di materiale ortotropo e cominciamo studiando la singola lamina.

Le fibre hanno orientazione $\theta$. Il sistema di riferimento originale è $x, y$, quello rispetto all'ortotropia, quindi ruotato di   $\theta$, è $n, t$.

Per studiare la risposta del materiale:
\begin{enumerate}
    \item Lo carico nella direzione dell'ortotropia. Sollecitando lungo queste direzioni la risposta è simmetrica, senza accoppiamento tra sforzi e deformazioni ($\frac{\nu_{12}}{E_{22}}=\frac{\nu_{21}}{E_{11}}$).
    \begin{equation*}
\left\{
\begin{array}{c}
\varepsilon_{1} \\
\varepsilon_{2} \\
\gamma_{12}
\end{array}
\right\}
=
\left[
\begin{array}{ccc}
\displaystyle \frac{1}{E_{11}} & \displaystyle-\frac{\nu_{12}}{E_{22}} & 0\\
\displaystyle -\frac{\nu_{21}}{E_{11}} & \displaystyle\frac{1}{E_{22}} &0 \\
0& 0 & \displaystyle\frac{1}{G}
\end{array}
\right]\,
\left\{
\begin{array}{c}
\sigma_{1} \\
\sigma_{2} \\
\tau_{12}
\end{array}
\right\}
\end{equation*}
    \\
    \item Se carico in una direzione diversa nasce una deformazione con l'accoppiamento di allungamento e taglio, nascono degli scorrimenti.
\end{enumerate}

Il materiale è caratterizzato da 4 parametri:  $E_{11}$,$E_{22}$,  $\nu_{12}=\nu_{21}$,  $G$.

Una volta considerata la rotazione del legame elastico $[D] =  \left[T\right][\bar{D}]\left[T\right]^T$, passiamo all'impaccamento delle lamine, seguendo il libro di laminazione .\\ Sviluppiamo un modello di piastra composta da $N$ lamine. \\ Ogni lamina è caratterizzata dalla sua distanza dal piano medio.

Una piastra isotropa ha uno stato di deformazione continuo nello spessore, somma di traslazione e flessione. Le lamine però essendo ruotate hanno fra loro rigidezze completamente diverse, lo stato di sforzo risulta discontinuo nello spessore. Globalmente l'andamento è comunque la somma di una traslazione e di una flessione a farfalla.

Consideriamo che le azioni interne recepiscono il contributo di tutte le lamine, per cui:

\begin{align*}
    \left\{N\right\} &= \int_{-\frac{t}{2}}^{\frac{t}{2}}\,  \left\{\sigma\right\} \,dz\\
    &= \sum^N_{i=1}\int_{h_{i-1}}^{h_i}\,  \left\{\sigma\right\}_i \,dz\\
     &= \sum^N_{i=1}\left(\int_{h_{i-1}}^{h_i}\, [D]_i \left\{\varepsilon\right\}_0 \,dz       
     +\int_{h_{i-1}}^{h_i}\, [D]_i \,z\left\{k\right\} \,dz           \right)\\
\end{align*}
Notiamo che $ \left\{\varepsilon\right\}_0$ e $\left\{k\right\}  $ dipendono solo dallo spostamento del piano medio, possono quindi uscire da integrali e sommatorie:
\begin{align*}
    [A] &:= \sum^N_{i=1}\int_{h_{i-1}}^{h_i}\,  [D]_i \,dz\\
      [B] &:= \sum^N_{i=1}\int_{h_{i-1}}^{h_i}\,  [D]_i\,z \,dz
\end{align*}

\begin{align*}
    \left\{M\right\} &= \begin{Bmatrix}
        M_y\\M_x\\M_{xy}
    \end{Bmatrix} =\int_{-\frac{t}{2}}^{\frac{t}{2}}\,  \left\{\sigma\right\} z\,dz\\
    &= \sum^N_{i=1}\int_{h_{i-1}}^{h_i}\,  \left\{\sigma\right\}_i z\,dz\\
     &= \sum^N_{i=1}\left(\int_{h_{i-1}}^{h_i}\, [D]_i \,z\left\{\varepsilon\right\}_0 \,dz       
     +\int_{h_{i-1}}^{h_i}\, [D]_i \,z^2\left\{k\right\} \,dz           \right)\\
\end{align*}

Da cui:
\begin{align*}
      [C] := \sum^N_{i=1}\int_{h_{i-1}}^{h_i}\,  [D]_i\,z^2 \,dz
\end{align*}
Rielaborando le matrici definite:
\begin{align*}
    [A] &:= \sum^N_{i=1}\,  [D]_i \,(h_i-h_{i-1})\\
      [B] &:= \frac{1}{2}\sum^N_{i=1}\,  [D]_i \,(h_i^2-h_{i-1}^2)\\
       [C] &:=\frac{1}{3} \sum^N_{i=1}\,  [D]_i \,(h_i^3-h_{i-1}^3)\\
\end{align*}

Il sistema completo diviene:

\begin{equation*}
\left\{
\begin{array}{c}
N\\M
\end{array}
\right\}
=
\begin{bmatrix}
[A] & [B] \\
[B] & [C]
\end{bmatrix}
\left\{
\begin{array}{c}
\boldsymbol{\varepsilon_0}\\
\boldsymbol{k}
\end{array}
\right\}
\end{equation*}

La matrice risultante è completamente piena, tirando il laminato questo si torce, si flette e si deforma a taglio.\\Il legame si può invertire:
\begin{equation*}
\left\{
\begin{array}{c}
\boldsymbol{\varepsilon_0}\\
\boldsymbol{k}
\end{array}
\right\}= \begin{bmatrix}
[A] & [B] \\
[B] & [C]
\end{bmatrix}^{-1}
\left\{
\begin{array}{c}
N\\M
\end{array}
\right\}
\end{equation*}

Per verificare i valori di sforzo ammissibili: calcolo le deformazioni (diverse) e poi gli sforzi, nelle facce inferiore e superiore di ogni lamina, e le giudico con il criterio scelto.