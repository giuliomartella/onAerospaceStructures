\section{Piastra di Kirchhoff. Deformazione flessionale della piastra omogenea isotropa.}

Partiamo dal legame momento-curvatura:
\begin{equation*}
\left\{
\begin{array}{c}
M_y \\
M_x\\
M_{xy}
\end{array}
\right\} =
\left[
\begin{array}{ccc}
\displaystyle \bar{D} & \displaystyle \nu\bar{D} & \displaystyle 0\\
\displaystyle \nu\bar{D} & \displaystyle \bar{D} & \displaystyle 0 \\
\displaystyle 0& \displaystyle 0 & \displaystyle \bar{D}\frac{1-\nu}{2}
\end{array}
\right]\,
\left\{
\begin{array}{c}
-w_{/xx} \\
-w_{/yy}\\
-2w_{/xy}
\end{array}
\right\} 
\end{equation*}

Applichiamo $M_x=\bar{M_x}$, gli altri momenti sono posti a zero $M_y,M_{xy}=0$

\begin{align*}
    M_y&=-\bar{D}w_{/xx}-\bar{D}w_{/yy}=0\qquad \rightarrow\qquad w_{/yy}=-w_{/xx}\frac{1}{\nu}\\
    M_x&=-\bar{D}\nu \,w_{/xx}-\bar{D}w_{/yy}=  -\bar{D}\nu \,w_{/xx}+\bar{D}w_{/xx}\frac{1}{\nu}\\
\end{align*}
\begin{align*}
   w_{/xx}&=\frac{\bar{M_x}}{\bar{D}\left(\frac{1}{\nu}-\nu\right)}\\
   w_{/yy}&=-\frac{\bar{M_x}}{\nu\bar{D}\left(\frac{1}{\nu}-\nu\right)}
\end{align*}

Quindi un momento $M_x$ genera anche una flessione, di segno opposto, anche lungo $y$. La flessione è detta anticlastica, quindi con due curvature di segno opposto lungo le due direzioni principali.\\
Per un materiale classico, con $0<\nu<1 $, la sella è più pronunciata in $x$ che in $y$. Questo comportamento è inedito rispetto al modello di trave.


