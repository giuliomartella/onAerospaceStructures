\section{Piastra di Kirchhoff. Legame spostamenti risultanti forze e momenti.}

Le azioni interne per la piastra sono:
\begin{equation*}
    \left\{
\begin{array}{c}
N_x \\
N_y\\
N_{xy}
\end{array}
\right\} = \int_{-\frac{t}{2}}^{\frac{t}{2}}\,  \left\{\sigma\right\} \,dz
=  \int_{-\frac{t}{2}}^{\frac{t}{2}}\, [D] \left\{\varepsilon_0\right\} \,dz
 +\int_{-\frac{t}{2}}^{\frac{t}{2}}\,z [D] \left\{k\right\} \,dz
 =[A]   \left\{\varepsilon_0\right\}  +[B] \left\{k\right\}
\end{equation*}

\begin{equation*}
    \left\{
\begin{array}{c}
M_y \\
M_x\\
M_{xy}
\end{array}
\right\} = \int_{-\frac{t}{2}}^{\frac{t}{2}}\,  z\left\{\sigma\right\} \,dz
=  \int_{-\frac{t}{2}}^{\frac{t}{2}}\, z[D] \left\{\varepsilon_0\right\} \,dz
 +\int_{-\frac{t}{2}}^{\frac{t}{2}}\,z^2 [D] \left\{k\right\} \,dz
 =[B]   \left\{\varepsilon_0\right\}  +[C] \left\{k\right\}
\end{equation*}

In un unica notazione:
\begin{equation*}
\left\{
\begin{array}{c}
N_x \\
N_y\\
N_{xy}\\
M_y \\
M_x\\
M_{xy}
\end{array}
\right\}
=
\begin{bmatrix}
[A] & [B] \\
[B] & [C]
\end{bmatrix}
\left\{
\begin{array}{c}
\boldsymbol{\varepsilon_0}\\
\boldsymbol{k}
\end{array}
\right\}
\end{equation*}

Per una piastra omogenea ed isotropa $[B]=0$. I termini sono disaccoppiati, le forze dipendono solo da $\boldsymbol{\varepsilon_0}$ e i momenti solo da $\boldsymbol{k}$.
\begin{equation*}
\left\{
\begin{array}{c}
\boldsymbol{N}\\
\boldsymbol{M}
\end{array}
\right\}
=
\begin{bmatrix}
[A] & \boldsymbol{0} \\
\boldsymbol{0} & [C]
\end{bmatrix}
\left\{
\begin{array}{c}
\boldsymbol{\varepsilon_0}\\
\boldsymbol{k}
\end{array}
\right\}
\end{equation*}

Al contrario per piastre che non sono omogenee ed isotrope i termini sono accoppiati, per esempio tirandola o applicando un taglio questa si può anche flettere.

Tornando alle ipotesi predenti e quindi con $[B]=0$, consideriamo i momenti:


\begin{equation*}
    \left\{
\begin{array}{c}
M_y \\
M_x\\
M_{xy}
\end{array}
\right\} = \int_{-\frac{t}{2}}^{\frac{t}{2}}\,  z\left\{\sigma\right\} \,dz
 =\int_{-\frac{t}{2}}^{\frac{t}{2}}\,z^2 [D] \left\{k\right\} \,dz
\qquad \qquad [D]=
\left[
\begin{array}{ccc}
\displaystyle 1 & \displaystyle \nu & \displaystyle 0\\
\displaystyle \nu & \displaystyle 1 & \displaystyle 0 \\
\displaystyle 0& \displaystyle 0 & \displaystyle \frac{1-\nu}{2}
\end{array}
\right]\,
\frac{E}{1-\nu^2}
\end{equation*}

\begin{equation*}
   M_y =  \int_{-\frac{t}{2}}^{\frac{t}{2}}\,z^2 
   \left[\begin{array}{c}
1\quad\nu\quad0
\end{array} \right] 
   \left[
   \begin{array}{c}
-w_{/xx} \\-w_{/yy}\\-2w_{/xy}
\end{array} \right] \frac{E}{1-\nu^2}\,dz
=(-w_{/xx} -\nu \,w_{/yy}) \frac{E}{1-\nu^2}\frac{t^3}{12}
\end{equation*}

Definiamo $\bar{D}=\frac{E}{1-\nu^2}\frac{t^3}{12}$:
\begin{align*}
     M_y &= -\bar{D}\,w_{/xx} -\nu \bar{D}\,w_{/yy}\\
     M_x &= -\nu \bar{D}\,w_{/xx} - \bar{D}\,w_{/yy}\\
     M_{xy} &= -(1-\nu)\bar{D}\,w_{/xy} 
\end{align*}

Ritroviamo quindi la matrice $[D]$, ovvero il legame fra i momenti e la curvatura:
\begin{equation*}
\left\{
\begin{array}{c}
M_y \\
M_x\\
M_{xy}
\end{array}
\right\} =
\left[
\begin{array}{ccc}
\displaystyle \bar{D} & \displaystyle \nu\bar{D} & \displaystyle 0\\
\displaystyle \nu\bar{D} & \displaystyle \bar{D} & \displaystyle 0 \\
\displaystyle 0& \displaystyle 0 & \displaystyle \bar{D}\frac{1-\nu}{2}
\end{array}
\right]\,
\left\{
\begin{array}{c}
-w_{/xx} \\
-w_{/yy}\\
-2w_{/xy}
\end{array}
\right\} 
\end{equation*}

Analogamente per le forze:

\begin{equation*}
\left\{
\begin{array}{c}
N_x \\
N_y\\
N_{xy}
\end{array}
\right\} =
\left[
\begin{array}{ccc}
\displaystyle 1 & \displaystyle \nu & \displaystyle 0\\
\displaystyle \nu & \displaystyle 1 & \displaystyle 0 \\
\displaystyle 0& \displaystyle 0 & \displaystyle \frac{1-\nu}{2}
\end{array}
\right]\, \frac{tE}{1-\nu^2}
\left\{
\begin{array}{c}
u_{0/x} \\
v_{0/y}\\
u_{0/y}+u_{0/x}
\end{array}
\right\} 
\end{equation*}








