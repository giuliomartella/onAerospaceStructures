\section{Elementi finiti isoparametrici.}

\begin{definizioneBox}
    Negli elementi finiti isoparametrici, in meccanica strutturale, le stesse funzioni di forma vengono utilizzate per descrivere sia la geometria dell'elemento sia il campo di spostamenti.
La matrice di rigidezza non richiede una rotazione del sistema di riferimento, poiché l'orientamento dell'elemento è già considerato nel processo di formazione della matrice stessa.
\end{definizioneBox}

La matrice $[K]_i$ è composta da una parte indipendente dalle deformazioni e da una parte dipendente dalle deformazioni dell'elemento (è una parte molto piccola). 

\begin{equation*}
    \left\{u \right\}_j = \begin{Bmatrix}
        u_x\\u_y\\u_z 
    \end{Bmatrix}= [H(x,y,z)] \,\left\{u_i \right\}_j
\end{equation*}

\subsubsection*{Elemento monodimensionale con due nodi}
 Consideriamo le coordinate globali $x,y,z$ e le coordinate isoparametriche $r,s,t$.\\
 Il sistema di riferimento isoparametrico è indipendente dalla dimensione dell'elemento (allineato con $r$), si ha sempre il primo nodo in $r=-1$ e il secondo in $r=1$ (questo concetto è estendibile a più dimensioni).
 
La posizione di un qualsiasi punto si può esprimere, nel caso monodimensionale, come:
\begin{equation*}
    x=\frac{1}{2}(1-r)x_1+\frac{1}{2}(1+r)x_2
\end{equation*}
Ovvero,
\begin{equation*}
    x=\sum^2_{i=1} h_ix_i \qquad \text{con} \quad h_1=\frac{1}{2}(1-r), \quad  h_2=\frac{1}{2}(1+r)
\end{equation*}
In forma matriciale:
\begin{equation*}
    \left\{x \right\} = \left[\frac{1}{2}(1-r) \quad  \frac{1}{2}(1+r)\right]\begin{Bmatrix}
        x_1\\x_2
    \end{Bmatrix}= [H] \,\left\{x_i \right\}
\end{equation*}

Vale lo stesso per gli spostamenti:
\begin{equation*}
    u=\sum^2_{i=1} h_iu_i \qquad \text{con} \quad h_1=\frac{1}{2}(1-r), \quad  h_2=\frac{1}{2}(1+r)
\end{equation*}
\begin{equation*}
    \left\{u \right\} = \left[\frac{1}{2}(1-r) \quad  \frac{1}{2}(1+r)\right]\begin{Bmatrix}
        u_1\\u_2
    \end{Bmatrix}= [H] \,\left\{u_i \right\}
\end{equation*}





