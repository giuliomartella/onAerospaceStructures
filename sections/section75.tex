\section{Elementi finiti isoparametrici.}

\begin{definizioneBox}
    Negli elementi finiti isoparametrici, in meccanica strutturale, le stesse funzioni di forma vengono utilizzate per descrivere sia la geometria dell'elemento sia il campo di spostamenti.
La matrice di rigidezza non richiede una rotazione del sistema di riferimento, poiché l'orientamento dell'elemento è già considerato nel processo di formazione della matrice stessa.
\end{definizioneBox}

La matrice $[K]_i$ è composta da una parte indipendente dalle deformazioni e da una parte dipendente dalle deformazioni dell'elemento (è una parte molto piccola). 

\begin{equation*}
    \left\{u \right\}_j = \begin{Bmatrix}
        u_x\\u_y\\u_z 
    \end{Bmatrix}= [H(x,y,z)] \,\left\{u_i \right\}_j
\end{equation*}

\subsubsection*{Elemento monodimensionale a due nodi}
 Consideriamo le coordinate globali $x,y,z$ e le coordinate isoparametriche $r,s,t$.\\
 Il sistema di riferimento isoparametrico è indipendente dalla dimensione dell'elemento (allineato con $r$), si ha sempre il primo nodo in $r=-1$ e il secondo in $r=1$ (questo concetto è estendibile a più dimensioni).
 
La posizione di un qualsiasi punto si può esprimere, nel caso monodimensionale, come:
\begin{equation*}
    x=\frac{1}{2}(1-r)x_1+\frac{1}{2}(1+r)x_2
\end{equation*}
Ovvero,
\begin{equation*}
    x=\sum^2_{i=1} h_ix_i \qquad \text{con} \quad h_1=\frac{1}{2}(1-r), \quad  h_2=\frac{1}{2}(1+r)
\end{equation*}
In forma matriciale:
\begin{equation*}
    \left\{x \right\} = \left[\frac{1}{2}(1-r) \quad  \frac{1}{2}(1+r)\right]\begin{Bmatrix}
        x_1\\x_2
    \end{Bmatrix}= [H] \,\left\{x_i \right\}
\end{equation*}

Vale lo stesso per gli spostamenti:
\begin{equation*}
    u=\sum^2_{i=1} h_iu_i \qquad \text{con} \quad h_1=\frac{1}{2}(1-r), \quad  h_2=\frac{1}{2}(1+r)
\end{equation*}
\begin{equation*}
    \left\{u \right\} = \left[\frac{1}{2}(1-r) \quad  \frac{1}{2}(1+r)\right]\begin{Bmatrix}
        u_1\\u_2
    \end{Bmatrix}= [H] \,\left\{u_i \right\}
\end{equation*}

Valutiamo le deformazioni considerando che gli spostamenti ammessi sono solo in $x$:
\begin{equation*}
    \varepsilon_{xx}=u_{x/x}=u_{/x}=\frac{du}{dr}\frac{dr}{dx}
\end{equation*}

Il vantaggio degli isoparametrici è che posso usare la stessa funzione di forma sia per $x$ che per $u$.
\begin{align*}
    \frac{dx}{dr}&=\frac{x_2}{2}-\frac{x_1}{2}=\frac{L}{2}  \qquad\rightarrow\qquad dx=\frac{L}{2}dr\\
      \frac{du}{dr}&=\frac{u_2-u_1}{2}
\end{align*}
La quantità $\frac{L}{2}$ corrisponde al determinante dello jacobiano. Cambiando lo jacobiano possiamo calcolare la matrice di rigidezza di geometrie diverse.

\begin{equation*}
    \left\{\varepsilon \right\}_j = [\mathcal{D}][H] \,\left\{u \right\}= [B] \,\left\{u \right\}
\end{equation*}
\begin{align*}
  [K]  &=  \int_\Omega [B]^T[D][B]dV\\
   &=  \int_L \bar{A}[B]^T[D][B]dx\\
    &=  \int_{-1}^1\bar{A}E\frac{1}{L}   \begin{Bmatrix}
        -1\\1\end{Bmatrix} [-1\quad 1] \frac{1}{L}\frac{L}{2}
    dx\\
    &= \frac{ \bar{A}E }{L} \begin{bmatrix}
        1&-1\\-1&1
    \end{bmatrix}
\end{align*}


\subsubsection*{Elemento monodimensionale a tre nodi}

Le funzioni di forma lineari con due nodi con tre ora diventano paraboliche. Se il secondo nodo è centrale le funzioni di forma sono:
\begin{align*}
    h_1&=\frac{1}{2}(1-r)-\frac{1}{2}(1-r^2)\\
    h_2&=\frac{1}{2}(1+r)-\frac{1}{2}(1-r^2)\\
    h_3&= (1-r^2)
\end{align*}

Il numero di nodi è $q$, le coordinate di ogni punto sono:
\begin{align*}
    x&=\sum^q_{i=1}h_ix_i\\
      y&=\sum^q_{i=1}h_iy_i\\
        z&=\sum^q_{i=1}h_iz_i\\
\end{align*}

\subsubsection*{Elemento piano a quattro nodi}


Le funzioni di forma sono:

\begin{align*}
    h_1&=\frac{1}{4}(1+r)(1+s)\\
    h_2&=\frac{1}{4}(1-r)(1+s)\\
    h_3&=\frac{1}{4}(1-r)(1-s)\\
    h_4&=\frac{1}{4}(1+r)(1-s)\\
\end{align*}

Dove :
\begin{align*}
 \left\{x \right\} &= [H] \,\left\{x_i \right\}\\
    \left\{u \right\} &= [H] \,\left\{u_i \right\}\\
    \begin{Bmatrix}
        x\\y\\z
    \end{Bmatrix}&=
    \left[\begin{array}{cccccccccccc}
        h_1 & h_2 & h_3 & h_4 & 0 & 0 & 0 & 0 & 0 & 0 & 0 & 0\\
        0 & 0 & 0 & 0 & h_1 & h_2 & h_3 & h_4 & 0 & 0 & 0 & 0\\
        0 & 0 & 0 & 0 & 0 & 0 & 0 & 0 & h_1 & h_2 & h_3 & h_4
    \end{array}\right]
    \begin{Bmatrix}
        x_1\\x_2\\x_3\\x_4\\y_1\\y_2\\y_3\\y_4\\z_1\\z_2\\z_3\\z_4
    \end{Bmatrix}\\
    \begin{Bmatrix}
        u_x\\u_y\\u_z
    \end{Bmatrix}&=
    \left[\begin{array}{cccccccccccc}
        h_1 & h_2 & h_3 & h_4 & 0 & 0 & 0 & 0 & 0 & 0 & 0 & 0\\
        0 & 0 & 0 & 0 & h_1 & h_2 & h_3 & h_4 & 0 & 0 & 0 & 0\\
        0 & 0 & 0 & 0 & 0 & 0 & 0 & 0 & h_1 & h_2 & h_3 & h_4
    \end{array}\right]
    \begin{Bmatrix}
        u_{1x}\\u_{2x}\\u_{3x}\\u_{4x}\\u_{1y}\\u_{2y}\\u_{3y}\\u_{4y}\\u_{1z}\\u_{2z}\\u_{3z}\\u_{4z}
    \end{Bmatrix}
\end{align*}
Abbiamo quindi un legame che definisce la geometria e gli spostamenti, anche in due dimensioni.
 
\begin{align*}
    x&=\sum^q_{i=1}h_ix_i\qquad &y&=\sum^q_{i=1}h_iy_i\qquad  & z&=\sum^q_{i=1}h_iz_i\\
     u_x&=\sum^q_{i=1}h_iu_{xi}\qquad &u_y&=\sum^q_{i=1}h_iu_{yi}\qquad  & u_z&=\sum^q_{i=1}h_i u_{zi}
\end{align*}

Le deformazioni sono:

\begin{equation*}
    \varepsilon_{ij}=(u_{i/j} +u_{j/i})
\end{equation*}
Dove per le derivate valutiamo:
\begin{equation*}
    \frac{\partial }{\partial r}=   \frac{\partial }{\partial x}  \frac{\partial x}{\partial r}+  \frac{\partial }{\partial y}  \frac{\partial y}{\partial r}+
      \frac{\partial }{\partial z}  \frac{\partial z}{\partial r}
\end{equation*}
In forma matriciale:
\begin{align*}
    \begin{Bmatrix}
          \frac{\partial }{\partial r}\\  \frac{\partial }{\partial s}\\  \frac{\partial }{\partial t}
    \end{Bmatrix}&=
\begin{bmatrix}
    \frac{\partial x}{\partial r} & \frac{\partial y}{\partial r} & \frac{\partial z}{\partial r}\\
    \frac{\partial x}{\partial s} & \frac{\partial y}{\partial s} & \frac{\partial z}{\partial s}\\
    \frac{\partial x}{\partial t} & \frac{\partial y}{\partial t} & \frac{\partial z}{\partial t}
\end{bmatrix}
     \begin{Bmatrix}
          \frac{\partial }{\partial x}\\  \frac{\partial }{\partial y}\\  \frac{\partial }{\partial z}
    \end{Bmatrix}\\
    \begin{bmatrix}
          \frac{\partial }{\partial r}
    \end{bmatrix}&=\begin{bmatrix}
          J
    \end{bmatrix} \begin{bmatrix}
          \frac{\partial }{\partial x}
    \end{bmatrix}
\end{align*}

Allora modifico l'integrazione: $[K]  =  \int_\Omega [B]^T[D][B]dV$
\begin{equation*}
    dV=dx\,dy\,dz=\det[J]\,dr\,ds\,dt
\end{equation*}

\subsubsection*{Esempi di calcolo dello jacobiano}
\begin{esempioBox}
    Elemento lineare a tre nodi.
    Le funzioni di forma, con un nodo al centro, sono:
    \begin{align*}
    h_1&=\frac{1}{2}(1-r)-\frac{1}{2}(1-r^2)=-\frac{r}{2}(1-r)\\
    h_2&=\frac{1}{2}(1+r)-\frac{1}{2}(1-r^2)=\frac{r}{2}(1+r)\\
    h_3&= (1-r^2)
\end{align*}
Gli spostamenti, in funzione delle precedenti:
\begin{equation*}
    \left\{u \right\} = \left[-\frac{r}{2}(1-r) \quad  \frac{r}{2}(1+r)\right]\begin{Bmatrix}
        u_1\\u_2\\u_3
    \end{Bmatrix}= [H] \,\left\{u_i \right\}
\end{equation*}
\begin{equation*}
    \frac{dH}{dr}=\left[-\frac{1}{2}+r\qquad\frac{1}{2}+r\qquad 2r \right]
\end{equation*}
Per cui le deformazioni sono:
\begin{equation*}
    \varepsilon=\frac{\partial u_x}{\partial x}=\frac{\partial u_x}{\partial r} \frac{\partial r}{\partial x}
\end{equation*}

In un generico punto dell'elemento, sapendo poi che le posizioni dei nodi sono $x_2=x_1+L$, $x_3=x_1+\frac{L}{2}$:
\begin{align*}
    x &= -\frac{r}{2}(1-r)x_1 + \frac{r}{2}(1+r)x_2 + (1-r^2)x_3\\
       &= -\frac{r}{2}(1-r)x_1 + \frac{r}{2}(1+r)(x_1+L) + (1-r^2)(x_1+\frac{L}{2})\\
       &= x_1+\frac{L}{2}+\frac{L}{2}r
\end{align*}

Per quanto riguarda lo jacobiano:
\begin{equation*}
    [J]= \left[\frac{\partial x}{\partial r} \right]=\frac{L}{2} \qquad \rightarrow\qquad \det J=\frac{L}{2}
\end{equation*}

La matrice di rigidezza diventa:
\begin{align*}
  [K]   &=  \int_L \bar{A}E[B]^T[B]  \det J dr\\
         &=  \int_L \bar{A}E[B]^T[B]  \frac{L}{2} dr\\  
    &=  \int \bar{A}E\   \begin{Bmatrix}
        -\frac{1}{2}+r\\ \frac{1}{2}+r \\ 2r \end{Bmatrix} \left[  -\frac{1}{2}+r\qquad \frac{1}{2}+r \qquad 2r \right]\frac{L}{2}
    dr\\
\end{align*}
    
\end{esempioBox}

\begin{esempioBox}
    Consideriamo un elemento piano a 4 nodi che nel piano $x,y$ è simmetrico rispetto l'origine e ha come primo vertice $(3,2)$. Le funzioni di forma sono:
    \begin{align*}
    h_1&=\frac{1}{4}(1+r)(1+s)=\frac{1}{4}(1+s+r+rs)\\
    h_2&=\frac{1}{4}(1-r)(1+s)=\frac{1}{4}(1+s-r-rs)\\
    h_3&=\frac{1}{4}(1-r)(1-s)=\frac{1}{4}(1-s-r+rs)\\
    h_4&=\frac{1}{4}(1+r)(1-s)=\frac{1}{4}(1-s+r-rs)\\
\end{align*}

\begin{align*}
    x &= \frac{1}{4}\left[x_1(1+s+r+rs) + x_2(1+s-r-rs) + x_3(1-s-r+rs) + x_4(1-s+r-rs)\right]\\
    y &= \frac{1}{4}\left[y_1(1+s+r+rs) + y_2(1+s-r-rs) + y_3(1-s-r+rs) + y_4(1-s+r-rs)\right]
\end{align*}

Sostituisco i valori dei vertici:
\begin{align*}
    x_1&= 3 \qquad &x_2&=-3 \qquad &x_3&=-3 \qquad &x_4&=3\\
     y_1&= 2 \qquad &y_2&=2 \qquad &y_3&=-2 \qquad &y_4&=-2\\
\end{align*}
Per cui:
\begin{align*}
    x&=3r\\
    y&=2s
\end{align*}
Nella jacobiana:
\begin{align*}
    J = \begin{bmatrix}
        \frac{\partial x}{\partial r} & \frac{\partial y}{\partial r}\\
        \frac{\partial x}{\partial s} & \frac{\partial y}{\partial s}
    \end{bmatrix} = 
    \begin{bmatrix}
        3 & 0\\
        0 & 2
    \end{bmatrix}
\end{align*}

Infine il suo determinante:
\begin{align*}
    \det(J) = 3 \cdot 2 - 0 \cdot 0 = 6
\end{align*}





\end{esempioBox}
