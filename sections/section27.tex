\section{Modello di trave. Calcolo matrice di rigidezza per trave libera con metodo agli spostamenti}

Partiamo dall'espressione generica della matrice di rigidezza in notazione di Voigt, consideriamo il solito problema di una trave incastrata.


\begin{equation*}
\left\{
\begin{array}{c}
F_{11}\\
F_{12} \\
F_{13} \\
F_{21} \\
F_{22} \\
F_{23}
\end{array}
\right\}
=
\begin{bmatrix}
\mathbf{A} & \mathbf{B} \\
\mathbf{C} & \mathbf{D}
\end{bmatrix}
\left\{
\begin{array}{c}
s_{11} \\
s_{12} \\
s_{13} \\
s_{21} \\
s_{22} \\
s_{23}
\end{array}
\right\}
\end{equation*}

Usiamo la scomposizione della matrice di rigidezza che abbiamo definito per dividere e semplificare il problema. Partiamo dalla matrice $D$ che, quando gli spostamenti sul primo estremo sono nulli (è presente un incastro), lega le forze sul secondo estremo agli spostamenti sul secondo estremo.\\
\begin{esempioBox}
Per calcolare $D_{11}$ risolviamo la struttura con il metodo degli spostamenti e il PLV. Sappiamo che $F_{21}=D_{11}s_{21}$. Ammesso che l'asse $z$ parta dall'incastro assumiamo per lo spostamento assiale una forma lineare $w_z = az+b$. Dopodiché per congruenza rispetto ai vincoli scriviamo:
\begin{equation*}
    \begin{cases}
        w_z(0 )=0\\
         w_z(l )=s_{21}\\
    \end{cases}
    \quad\quad\quad \rightarrow\quad\quad\quad
        \begin{cases}
            b=0\\
            a = \frac{s_{21}}{l}
        \end{cases}
      \quad\quad\quad \rightarrow\quad\quad\quad
    w_z(z) = \frac{s_{21}}{l}z
\end{equation*}

Adesso scriviamo l'espressione del PLV considerando che $T_z = w'_z \,EA = aEA $.
\begin{align*}
    \delta L_e\,&=\,F_{21}\, \delta s_{21}\\
    \delta L_d\,&=\,\int_l \delta w'_z \,EAw'_z\,dz\, = \,\int_l \delta s_{21} \frac{1}{l}\,EAs_{21} \frac{1}{l}\,dz\, 
    = \,\delta s_{21} \frac{EA}{l}s_{21} \\
    F_{21} \delta s_{21} \,&= \,\delta s_{21} \frac{EA}{l}s_{21} 
    \quad\quad\quad \rightarrow\quad\quad\quad
     F_{21}  \,= \, \frac{EA}{l}s_{21} \\
     D_{11} \,&= \, \frac{EA}{l}
\end{align*}
\end{esempioBox}

La matrice $\boldsymbol{B}$ si ricava considerando che le forze generalizzate sul primo estremo sono equilibranti rispetto alle forze sul secondo estremo. Scrivo l'equilibrio:

\begin{equation*}
    \begin{cases}
        F_{11} + F_{21} =0\\
        F_{12} + F_{22} =0\\
        F_{13} + F_{23} +F_{23}l =0\\
    \end{cases}
    \quad\quad\quad\rightarrow\quad\quad\quad
    \begin{cases}
        F_{11}   =-F_{21}=   -\frac{EA}{l}s_{21}\\
        F_{12}   =-F_{22}= -   \frac{12EJ}{l^3} s_{22}  +   \frac{6EJ}{l^2} s_{23}\\
        F_{13}  =- F_{23} -F_{23}l= -   \frac{6EJ}{l^2} s_{22}  +   \frac{2EJ}{l} s_{23} \\
    \end{cases}
\end{equation*}

La matrice $\boldsymbol{C}$ si trova per simmetria, $\boldsymbol{B} =\boldsymbol{C} $. La matrice $\boldsymbol{A}$ infine si ricava scrivendo le condizioni di equilibrio rispetto alla matrice $\boldsymbol{C}$.





