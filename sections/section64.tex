\section{Piastra di Kirchhoff. Legame spostamenti carichi distribuiti.}


Per la trave si può integrare il taglio per ottenere il momento flettente ma, nella piastra, non ho una $\tau_y$ verticale da integrare.\\
Devo comunque ipotizzare una risultante in direzione $ z$ per poter legare la pressione con il momento. Dunque non considero una forza per unità di area $\tau$ ma solo le sue risultanti $Q_x$, $Q_y$ che generano momento.

$M_x$ agisce su una faccia con normale diretta come $y$,  $M_y$ agisce su una faccia con normale diretta come $x$, $M_{xy}$ è generato dalle $\sigma_{xy}$ variabili nello spessore.

Consideriamo gli equilibri per una porzione $dx\times dy \times dz$:
\begin{itemize}
    \item $R_z=0$
    \begin{align*}
        (Q_x+Q_{x/x}dx)dy-Q_xdy + (Q_y+Q_{y/y}dy)dx-Q_ydx + pdxdy=0\\
        Q_{x/x}+Q_{y/y}+p=0
    \end{align*}        \\
    
      \item $M_x=0$
    \begin{align*}
       M_xdx- (M_x+M_{x/y}dy)dx + M_{xy}dy- (M_{xy}+M_{xy/x}dx)dy + Q_ydxdy=0\\
       M_{x/y}+M_{xy/x}-Q_y=0
    \end{align*}        \\

     \item $M_y=0$
    \begin{align*}
       M_ydy- (M_y+M_{y/x}dx)dy + M_{xy}dx- (M_{xy}+M_{xy/y}dy)dx + Q_xdxdy=0\\
       M_{y/x}+M_{xy/y}-Q_x=0
    \end{align*}        \\
\end{itemize}

Dunque:
\begin{equation*}
    \begin{cases}
         Q_{x/x}+Q_{y/y}+p=0\\ M_{x/y}+M_{xy/x}-Q_y=0\\ M_{y/x}+M_{xy/y}-Q_x=0
    \end{cases}
\end{equation*}
Ricombinandole:
\begin{align*}
    Q_y&=M_{x/y}+M_{xy/x}\\
    Q_x&=M_{y/x}+M_{xy/y}\\
    M_{y/xx}&+M_{xy/yx}+M_{x/yy}+M_{xy/xy}+p=0\\
    M_{x/yy}&+M_{y/xx}+2M_{xy/xy}=-p\\
\end{align*}

Vediamo come le forze $Q$ effettivamente non esistano, non esistono sforzi in tale direzione, servono comunque per collegare i momenti alla pressione. In forma matriciale:

\begin{equation*}
\begin{bmatrix} 
\frac{\partial^2}{\partial y^2} & \frac{\partial^2}{\partial x^2} & 2\frac{\partial^2}{\partial x \partial y} 
\end{bmatrix} 
\begin{bmatrix} 
M_x \\ 
M_y \\ 
M_{xy} 
\end{bmatrix} = -p
\end{equation*}

\begin{align*}
    &\begin{bmatrix} 
\frac{\partial^2}{\partial y^2} & \frac{\partial^2}{\partial x^2} & 2\frac{\partial^2}{\partial x \partial y} 
\end{bmatrix} 
\left[
\begin{array}{ccc}
\displaystyle \bar{D} & \displaystyle \nu\bar{D} & \displaystyle 0\\
\displaystyle \nu\bar{D} & \displaystyle \bar{D} & \displaystyle 0 \\
\displaystyle 0& \displaystyle 0 & \displaystyle \bar{D}\frac{1-\nu}{2}
\end{array}
\right]\,
\left\{
\begin{array}{c}
-w_{/xx} \\
-w_{/yy}\\
-2w_{/xy}
\end{array}
\right\}  = -p\\
&\begin{bmatrix} 
\frac{\partial^2}{\partial y^2} & \frac{\partial^2}{\partial x^2} & 2\frac{\partial^2}{\partial x \partial y} 
\end{bmatrix} 
\left[
\begin{array}{ccc}
\displaystyle 1 & \displaystyle \nu & \displaystyle 0\\
\displaystyle \nu& \displaystyle1 & \displaystyle 0 \\
\displaystyle 0& \displaystyle 0 & \displaystyle \frac{1-\nu}{2}
\end{array}
\right]\,
\left\{
\begin{array}{c}
w_{/xx} \\
w_{/yy}\\
2w_{/xy}
\end{array}
\right\}  = \frac{p}{\bar{D} }\\
&\begin{bmatrix} 
\frac{\partial^2}{\partial y^2} & \frac{\partial^2}{\partial x^2} & 2\frac{\partial^2}{\partial x \partial y} 
\end{bmatrix} 
\left\{
\begin{array}{c}
w_{/xx}+\nu w_{/yy} \\
\nu w_{/xx} +w_{/yy}\\
(1-\nu)w_{/xy}
\end{array}
\right\}  = \frac{p}{\bar{D} }
\end{align*}

Infine esplicitando le derivate:
\begin{equation*}
w_{/xxxx} + 2w_{/xxyy} + w_{/yyyy} = \frac{p}{\bar{D}}
\end{equation*}
