\section{Piastra di Mindlin.}

Il modello di Kirchhoff faceva uso delle forze $Q_x$ e $Q_y$ pur considerandole nulle. Mindlin nel suo modello supera questa contraddizione riuscendo a calcolarle.

Rispetto al modello precedente aggiungiamo le due rotazioni $\theta_x$ e $\theta_y$. Sono le rotazioni del piano medio attorno ai rispettivi assi. La sezione della piastra non rimane quindi ortogonale al piano medio.

Scriviamo lo spostamento del piano medio:
\begin{equation*}
\left\{
\begin{array}{c}s \end{array}
\right\} =
\left\{
\begin{array}{c}
u_0\\v_0\\w\\\theta_x\\\theta_y
\end{array}
\right\} 
\end{equation*}

Gli spostamenti in $x$ e $y$ diventano:
    \begin{align*}
        s_x&=u_0+z\theta_y\\
        s_y&=v_0-z\theta_x\\
    \end{align*}
Dunque le deformazioni:


\begin{equation*}
  \left\{\varepsilon\right\}=  \left\{\varepsilon_0\right\}+ z\left\{ k\right\}
\end{equation*}
Dove
\begin{equation*}
  \left\{ k\right\}=  \left\{
\begin{array}{c} \theta_{y/x}\\- \theta_{x/y}\\ \theta_{y/y}-\theta_{x/x}
\end{array}\right\}
\end{equation*}
Esplicitamente:

\begin{align*}
    \varepsilon_{xx}&=s_{x/x}= u_{0/x}+z\,\theta_{y/x}\\
     \varepsilon_{yy}&=s_{y/y}=  v_{0/y}-z\,\theta_{x/y}\\
      \gamma_{xy}&=s_{x/y}+s_{y/x}=      u_{0/y}+z\,\theta_{y/y}   + v_{0/x}-z\,\theta_{x/x}\\
       \gamma_{xz}&=s_{x/z}+s_{z/x}=     (u_0+z\theta_y)_z   + w_{/x}= \theta_y+ w_{/x}\\
         \gamma_{yz}&=s_{y/z}+s_{z/y}=     (v_0-z\theta_x)_z   + w_{/y}= -\theta_x+ w_{/y}
\end{align*}

Le ultime due $\gamma_{xz}$ e $\gamma_{yz}$ sono costanti in $z$, non variano nello spessore.

Ora valutiamo le forze:

\begin{equation*}
    Q_x=\int_{-\frac{t}{2}}^{\frac{t}{2}} \tau_{xz}dz \qquad\qquad  Q_y=\int_{-\frac{t}{2}}^{\frac{t}{2}} \tau_{yz}dz 
\end{equation*}
Usando $\tau=G\gamma$, in forma matriciale:

\begin{equation*}
    \begin{Bmatrix}
        Q_x \\
        Q_y
    \end{Bmatrix}
    =
    \int_{-\frac{t}{2}}^{\frac{t}{2}}
    \left[
\begin{array}{ccc}
\displaystyle G & \displaystyle 0 \\
\displaystyle 0& \displaystyle G 
\end{array}
\right]
    \begin{Bmatrix}
        \tau_{xz} \\
        \tau_{yz}
    \end{Bmatrix}
    dz=
        \left[
\begin{array}{ccc}
\displaystyle Gt & \displaystyle 0 \\
\displaystyle 0& \displaystyle G t
\end{array}
\right]
    \begin{Bmatrix}
        \theta_y+ w_{/x} \\
        -\theta_x+ w_{/y}
    \end{Bmatrix}    
\end{equation*}

Quindi $\gamma$ e quindi $\tau$ non variano nello spessore. Questa è un'approssimazione, infatti le $\tau$ dall'equazione dei flussi sorgenti sono nulle sulla superficie e crescono verso il centro.\\
Introduciamo un fattore correttivo $k_s$ detto fattore di taglio.
\begin{equation*}
    \begin{Bmatrix}
        Q_x \\
        Q_y
    \end{Bmatrix}
   =k_s
        \left[
\begin{array}{ccc}
\displaystyle Gt & \displaystyle 0 \\
\displaystyle 0& \displaystyle G t
\end{array}
\right]
    \begin{Bmatrix}
        \theta_y+ w_{/x} \\
        -\theta_x+ w_{/y}
    \end{Bmatrix}
\end{equation*}

Imponiamo $k_s=\frac{5}{6}$, in questo modo otteniamo la stessa energia di deformazione a taglio e nel caso in cui le $\tau$ siano variabili nello spessore. In particolare $V_d$ è equivalente a quella calcolata con $\tau_{xz}$ e $\tau_{yz}$ paraboliche nello spessore.