\section{Vettori degli sforzi e delle deformazioni}

Partiamo dall'espressione del Principio dei Lavori Virtuali.
\begin{align*}
     & \int_{S_{\sigma}}  \delta u_j T_j \, dA  \, +\,\int_{\Omega} \delta u_{j}  b_j  \, dV \, = \, \int_{\Omega} \delta \varepsilon_{ij}  \sigma_{ij}  \, dV   \\
    &  \int_{S_{\sigma}} \delta \mathbf{u} \cdot \mathbf{T} \, dA 
    \,+\, \int_{\Omega} \delta \mathbf{u} \cdot \mathbf{b} \, dV 
    \,=\, \int_{\Omega} \delta \boldsymbol{\varepsilon} : \boldsymbol{\sigma} \, dV
\end{align*}


Esplicitiamo l'ultimo termine sfruttando la simmetria di $\boldsymbol{\sigma}$ e $\boldsymbol{\varepsilon}$.

\begin{align*}
     & \int_{\Omega} \delta \varepsilon_{ij} \, \sigma_{ij} \, dV \\
  =\; & \int_{\Omega} \big( \delta \varepsilon_{xx} \, \sigma_{xx}
     + \delta \varepsilon_{yy} \, \sigma_{yy}
     + \delta \varepsilon_{zz} \, \sigma_{zz} \\
     & \qquad + \delta \varepsilon_{xy} \, \sigma_{xy}
     + \delta \varepsilon_{yx} \, \sigma_{yx}
     + \delta \varepsilon_{xz} \, \sigma_{xz} \\
     & \qquad + \delta \varepsilon_{zx} \, \sigma_{zx}
     + \delta \varepsilon_{yz} \, \sigma_{yz}
     + \delta \varepsilon_{zy} \, \sigma_{zy} \big) \, dV \\
  =\; & \int_{\Omega} \big( \delta \varepsilon_{xx} \, \sigma_{xx}
     + \delta \varepsilon_{yy} \, \sigma_{yy}
     + \delta \varepsilon_{zz} \, \sigma_{zz} \\
     & \qquad + 2 \delta \varepsilon_{xy} \, \sigma_{xy}
     + 2 \delta \varepsilon_{xz} \, \sigma_{xz}
     + 2 \delta \varepsilon_{yz} \, \sigma_{yz} \big) \, dV
\end{align*}


Riorganizziamo  $\boldsymbol{\sigma}$ e $\boldsymbol{\varepsilon}$ in vettori con sole componenti indipendenti, usando la notazione di Voigt. Questa forma perde il significato tensoriale, è importante ricordare che entrambi rimangono tensori di secondo ordine.

\begin{equation*}
\left\{\sigma\right\} =
\left\{
\begin{array}{c}
\sigma_{xx} \\[6pt]
\sigma_{yy} \\[6pt]
\sigma_{zz} \\[6pt]
\sigma_{yz} \\[6pt]
\sigma_{xz} \\[6pt]
\sigma_{xy}
\end{array}
\right\}
\quad , \quad
\left\{\varepsilon\right\} =
\left\{
\begin{array}{c}
\varepsilon_{xx} \\[6pt]
\varepsilon_{yy} \\[6pt]
\varepsilon_{zz} \\[6pt]
2 \varepsilon_{yz} \\[6pt]
2 \varepsilon_{xz} \\[6pt]
2 \varepsilon_{xy}
\end{array}
\right\}
=
\left\{
\begin{array}{c}
\varepsilon_{xx} \\[6pt]
\varepsilon_{yy} \\[6pt]
\varepsilon_{zz} \\[6pt]
\gamma_{yz} \\[6pt]
\gamma_{xz} \\[6pt]
\gamma_{xy}
\end{array}
\right\}
\quad , \quad
\gamma_{ij} = 2 \varepsilon_{ij}
\end{equation*}

I tensori di secondo ordine scritti nella notazione di Voigt non possono essere direttamente ruotati, per trovare la relazione di rotazione rigida bisogna tornare alla formulazione matriciale.


Usando questa notazione riscrivo l'espressione per il lavoro di deformazione:

\begin{itemize}
    \item PLV 
    \begin{equation*}
      \delta L_d \, = \,  \int_{\Omega}  \left\{\delta\varepsilon\right\}^T  \left\{\sigma\right\}  \, dV
    \end{equation*}\\
    \item PLVC
    \begin{equation*}
       \delta L_d \, = \,  \int_{\Omega}  \left\{\varepsilon\right\}^T  \left\{\delta\sigma\right\}  \, dV
    \end{equation*}
\end{itemize}

Esistono alcuni casi notevoli in cui si possono ridurre gli elementi indipendenti considerati nella notazione:
\begin{itemize}
    \item Modello di trave: 
    $\left\{\sigma\right\} = \left\{
    \begin{array}{c}
    \sigma_{zz} \\ \sigma_{zx} \\ \sigma_{zy}
    \end{array}
    \right\}$

    \item Modello di piastra: 
    $\left\{\sigma\right\} = \left\{
    \begin{array}{c}
    \sigma_{xx} \\ \sigma_{yy} \\ \sigma_{zz}
    \end{array}
    \right\}$

    \item Stato di sforzo piano: 
    $\left\{\sigma\right\} = \left\{
    \begin{array}{c}
    \sigma_{xx} \\ \sigma_{yy} \\ \sigma_{xy}
    \end{array}
    \right\}$

    \item Stato di deformazione piana: 
    $\left\{\varepsilon\right\} = \left\{
    \begin{array}{c}
    \varepsilon_{xx} \\ \varepsilon_{yy} \\ \varepsilon_{xy}
    \end{array}
    \right\}$
\end{itemize}





