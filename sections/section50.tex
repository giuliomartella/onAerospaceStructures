\section{Flessione differenziale}

La flessione differenziale è il meccanismo con cui le sezioni aperte (se vincolate) riescono ad assorbire un momento torcente. Accade solo in presenza di un incastro o di un vincolo capace di sostenere la reazione vincolare.

Considero una sezione aperta semplice con tre pannelli e quattro correnti, applico il momento $M_z = +M$. Valuto il primo flusso con l'approccio classico.
\begin{equation*}
    q_1 \,=\, -\frac{T_y}{J_x}S'_x-\frac{T_x}{J_y}S'_y \,=\,0
\end{equation*}
L'approccio classico prevede flussi nulli. Provo a usare l'equilibrio sui due assi e alle rotazioni.

\begin{align*}
    R_x &=0 \quad\rightarrow\quad q_2=0\\
    R_y &=0 \quad\rightarrow\quad q_1=q_3\\
\end{align*}
\begin{equation*}
    M_1 = \sum2q\Omega = 2q_3\Omega_3\quad\rightarrow\quad q_3=\frac{M}{ab}=q_1
\end{equation*}
Dunque grazie all'equilibrio conosco i flussi, uso quindi l'equazione dei flussi sorgenti in prima forma:
\begin{equation*}
    \Phi = -\frac{dN}{dz}\quad\quad\quad N= \int_A\,\sigma_z\,dA
\end{equation*}
\begin{align*}
    \Phi \,&=\, +\frac{M}{ab}=\,-\frac{dN_1}{dz}\\
    \Phi \,&=\, -\frac{M}{ab}=\,-\frac{dN_2}{dz}\\
    \Phi \,&=\, +\frac{M}{ab}=\,-\frac{dN_3}{dz}\\
    \Phi \,&=\, -\frac{M}{ab}=\,-\frac{dN_4}{dz}\\
\end{align*}

Adesso integriamo la prima:
\begin{align*}
     dN_1\,&=\,-\frac{M}{ab}dz\\
    \int_{N_1(0)}^{N_1(z)}  dN_1\,&=\,\int_0^z-\frac{M}{ab}dz\\
    N_1(z) - N_1(0)  &=-\frac{M}{ab}z
\end{align*}

Essendo la trave libera per $z=L$, devo avere $N_1(L) = 0$, per cui:
\begin{align*}
    N_1(L) - N_1(0)  &=-\frac{M}{ab}z\\
    N_1(0) &= \frac{M}{ab}L\\
     N_1(z) &= -\frac{M}{ab}z+\frac{M}{ab}L \\
\end{align*}

Considerando gli altri correnti:
\begin{align*}
     N_2(z) &= +\frac{M}{ab}z-\frac{M}{ab}L \\
      N_3(z) &= -\frac{M}{ab}z+\frac{M}{ab}L \\
       N_4(z) &= +\frac{M}{ab}z-\frac{M}{ab}L \\
\end{align*}

Dai risultati si evince che i correnti 1 e 3 lavorano a trazione, mentre i correnti 2 e 4 lavorano a compressione. Nella deformata $q_1$ flette verso il basso mentre $q_3$ verso l'alto.\\
Quindi in una trave con sezione aperta l'applicazione di un momento torcente genera delle azioni assiali nei correnti. Le sezioni non rimangono piane e non siamo più nel modello di de Saint Venant.\\

Nel caso in cui avessi più di tre pannelli, alle condizioni di equilibrio bisognerebbe aggiungere delle equazioni di congruenza (più complicate non valendo più il modello di trave).