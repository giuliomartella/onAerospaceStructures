\section{Tensore gradiente di deformazione e tensore di Green-Lagrange}
\subsubsection*{Primo Tensore}
Considerando un corpo che si muove e si deforma nello spazio, indichiamo la mappatura di un punto con $\Psi$ tale che:

\begin{equation*}
\mathbf{x} =  \Psi(\mathbf{X}) 
\quad \quad 
\mathbf{x} = 
\begin{bmatrix}
x_1 \\
x_2 \\
x_3
\end{bmatrix}
= 
\begin{bmatrix}
\psi_1(\mathbf{X}) \\
\psi_2(\mathbf{X}) \\
\psi_3(\mathbf{X})
\end{bmatrix}
\end{equation*}

Essendo che $\mathbf{u} + d\mathbf{x} = \Psi(\mathbf{X} + d\mathbf{X})$ e $\mathbf{u} = \Psi(\mathbf{X})$, risulta che per spostamenti infinitesimi:

\begin{equation*}
d\mathbf{x} = \Psi(\mathbf{X} + d\mathbf{X}) - \Psi(\mathbf{X})
\end{equation*}

\begin{definizioneBox}
Tensore gradiente di deformazione 
\begin{equation*}
\boldsymbol{F} = \frac{d\mathbf{x}}{d\mathbf{X}} = \frac{\Psi(\mathbf{X} + d\mathbf{X}) - \Psi(\mathbf{X})}{d\mathbf{X}}
\end{equation*}
\end{definizioneBox}

Risulta quindi :
\begin{equation*}
d\mathbf{x} = \boldsymbol{F}\ d\mathbf{X} 
\quad \quad  
\boldsymbol{F} = 
\begin{bmatrix}
\frac{\partial x}{\partial X} & \frac{\partial x}{\partial Y} & \frac{\partial x}{\partial Z} \\
\frac{\partial y}{\partial X} & \frac{\partial y}{\partial Y} & \frac{\partial y}{\partial Z} \\
\frac{\partial z}{\partial X} & \frac{\partial z}{\partial Y} & \frac{\partial z}{\partial Z}
\end{bmatrix}
\end{equation*}

Essendo $\mathbf{u}$ lo spostamento del punto $P$ risulta:

\begin{equation*}
\mathbf{u} = \mathbf{x} - \mathbf{X} = \Psi(\mathbf{X}) -\mathbf{X}, \quad \quad \Psi(\mathbf{X}) = \mathbf{X}+ \mathbf{u}
\end{equation*}

Infine: 
\begin{equation*}
\boldsymbol{F} = \frac{\partial\Psi(\mathbf{X})}{\partial \mathbf{x} } = \frac{ \partial \mathbf{u} }{\partial \mathbf{X} } + \frac{ \partial\mathbf{X} }{ \partial\mathbf{X} } = \frac{ \partial \mathbf{u} }{\partial \mathbf{X} } + \boldsymbol{I} 
\end{equation*}

\subsubsection*{Secondo Tensore}

Definiamo:
\begin{itemize}
    \item $dS^2 = d\mathbf{X}^T  d\mathbf{X} $, in condizione indeformata
    \item $ds^2 = d\mathbf{x}^T  d\mathbf{x} $, in condizione deformata
\end{itemize}
Sottraendoli fra loro:
\begin{align*}
    ds^2 - dS^2 &= d\mathbf{x}^T  d\mathbf{x}  - d\mathbf{X}^T  d\mathbf{X}\\
    &=  d\mathbf{X}^T \boldsymbol{F}^T \boldsymbol{F} d\mathbf{X} - d\mathbf{X}^T  d\mathbf{X}\\
    &= d\mathbf{X}^T (\boldsymbol{F}^T \boldsymbol{F} - \boldsymbol{I}) d\mathbf{X}
\end{align*}

\begin{definizioneBox}
    Tensore di Green-Lagrange
    \begin{equation*}
\boldsymbol{E} = \frac{1}{2} \left(\boldsymbol{F}^T \boldsymbol{F} - \boldsymbol{I}\right)
\end{equation*}
\end{definizioneBox}

\subsubsection*{Terzo Tensore}

Rielaboro il tensore di Green-Lagrange
\begin{align*}
\boldsymbol{E} &= \frac{1}{2} \left[ \left( \frac{\partial \mathbf{u}}{\partial \mathbf{X}} + \boldsymbol{I} \right)^{\!\top} \left( \frac{\partial \mathbf{u}}{\partial \mathbf{X}} + \boldsymbol{I} \right) - \boldsymbol{I} \right] \\
&= \frac{1}{2} \left[ \left( \frac{\partial \mathbf{u}}{\partial \mathbf{X}} \right)^{\!\top} \frac{\partial \mathbf{u}}{\partial \mathbf{X}} + \left( \frac{\partial \mathbf{u}}{\partial \mathbf{X}} \right)^{\!\top} + \frac{\partial \mathbf{u}}{\partial \mathbf{X}} \right]
\end{align*}

In componenti:
\begin{align*}
E_{ij} &= \frac{1}{2} \left( \frac{\partial u_k}{\partial X_i} \frac{\partial u_k}{\partial X_j} + \frac{\partial u_j}{\partial X_i} + \frac{\partial u_i}{\partial X_j} \right) \\
&= \frac{1}{2} \left( u_{k/i} \, u_{k/j} + u_{j/i} + u_{i/j} \right) 
\end{align*}

Consideriamo spostamenti e rotazioni infinitesimi $ u_{k/i} \, u_{k/j}  = 0$.
\begin{definizioneBox}
Tensore delle Piccole Deformazioni
    \begin{equation*}
\epsilon_{ij} = \frac{1}{2} \left( u_{j/i} + u_{i/j} \right) 
\end{equation*}
\end{definizioneBox}

$\boldsymbol{\epsilon}$ è il tensore coniugato al tensore degli sforzi di Cauchy, come questo è simmetrico.







