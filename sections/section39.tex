\section{Schema a semiguscio e risoluzione di sezioni ad una cella}


Data una generica sezione a semiguscio con $n$ correnti e $m $ pannelli ho come incognite $n$ azioni assiali per i correnti e $m$ flussi di taglio nei pannelli.

Procediamo con:
\begin{enumerate}
    \item Numerazione arbitraria dei pannelli e scelta arbitraria del verso positivo dei flussi.\\
    \item Numerazione arbitraria dei correnti.\\
    \item Numerazione arbitraria delle celle e scelta del verso positivo delle loro rotazioni.\\
    \item Per usare l'equazione dei flussi sorgenti in seconda forma serve riferirsi agli assi principali d'inerzia, per identificarli:
    \begin{itemize}
        \item Considerando che solo i correnti hanno area, trovo il baricentro e centro lì un nuovo sistema di riferimento parallelo al primo.
            \begin{equation*}
                \boldsymbol{x}_{cg}=\frac{\sum_i^nA_i\boldsymbol{x}_i}{\sum_i^nA_i}
            \end{equation*}\\
        \item Rispetto al nuovo sistema calcolo i momenti d'inerzia e il momento centrifugo:
    \begin{equation*}
     J_{\tilde{x}} = \sum_i A_i \tilde{y}_i^2 \quad\quad\quad
     J_{\tilde{y}} = \sum_i A_i \tilde{x}_i^2 \quad\quad\quad
     J_{\tilde{x}\tilde{y}} = \sum_i A_i \tilde{x}_i \tilde{y}_i
    \end{equation*}\\
    \item Se $ J_{\tilde{x}\tilde{y}} = 0$ $(\tilde{x},\tilde{y})$ sono già assi principali d'inerzia. Altrimenti devono essere ruotati di una quantità $\alpha$:
    \begin{equation*}
        \alpha = \frac{1}{2} \arctan\left(\frac{2 J_{\tilde{x}\tilde{y}} }{J_{\tilde{x}} J_{\tilde{y}} }\right)
    \end{equation*}\\
    \item Per trovare gli assi principali d'inerzia posso sfruttare eventuali simmetrie fra i correnti.
    \end{itemize}
    \item Calcolo le $\sigma$ attraverso le azioni assali nei correnti:
    \begin{align*}
        \sigma = \frac{T_z}{\bar{A}} +  \frac{M_x}{J_x}y -\frac{M_y}{J_y}x \\
        N_i = \sigma_iA_i= \frac{T_z}{\bar{A}}A_i +  \frac{M_x}{J_x}S_{xi} -\frac{M_y}{J_y}S_{yi}
     \end{align*}\\
    \item Calcolo le $\tau$ attraverso i flussi, a seconda del numero di celle $N$:
    \begin{itemize}
        \item Se $N=0$ la sezione è aperta e la struttura è labile a torsione, ci sono meno incognite che condizioni di equilibrio (flussi).\\
        \item Se $N=1$ la sezione ha una cella, la struttura è isostatica e ci sono tante incognite quante condizioni di equilibrio.\\
        \item Se $N>1$ la sezione ha più celle, la struttura è iperstatica, ci sono più incognite che condizioni di equilibrio e il sistema andrà completato con delle condizioni di congruenza.
    \end{itemize}
\end{enumerate}


\begin{esempioBox}
    Sezioni ad una cella.\\
    Consideriamo una sezione a semiguscio a una cella, di forma rettangolare, con 6 correnti e 6 pannelli. Le aree dei correnti misurano $A$, lo spessore dei pannelli $t$ e la sezione ha dimensione $2a\times h$, dove $a$ è la lunghezza di ogni pannello orizzontale. Sono applicati un taglio al centro $T_y$ e un momento $M_z$. Per simmetria dei correnti il baricentro si trova al centro della sezione.\\
    Il momento d'inerzia è $J_x=6A\frac{h^2}{4}=\frac{3}{2}Ah^2$. Ci troviamo nel metodo delle forze quindi scriviamo prima le condizioni di equilibrio. \\
    Uso l'equazione dei flussi sorgenti in seconda forma per scrivere $n-1$ condizioni di equilibrio, ogni volta adattando la linea di contorno considerata, dopodiché chiudo il sistema imponendo l'equilibrio alle rotazioni.
    \begin{equation*}
       \Phi\, = -\, \frac{T_y}{J_x}S'_x  \,-\, \frac{T_x}{J_y}S'_y 
    \end{equation*}
    \begin{align*}
        q_1-q_6 \,&=\, -\frac{T_y}{J_x}\frac{Ah}{2}\,=\, - \frac{T_y}{3h}\\
         q_2-q_1 \,&=\, -\frac{T_y}{J_x}\frac{Ah}{2}\,=\, - \frac{T_y}{3h}\\
          q_4-q_2 \,&=\, -\frac{T_y}{J_x}\left(\frac{Ah}{2}-\frac{Ah}{2}\right)\,=\, 0\\
           q_4-q_3 \,&=\, -\frac{T_y}{J_x}\left(-\frac{Ah}{2}\right)\,=\,  \frac{T_y}{3h}\\
            q_5-q_4 \,&=\, -\frac{T_y}{J_x}\left(-\frac{Ah}{2}\right)\,=\,  \frac{T_y}{3h}\\
    \end{align*}
    Per l'equilibrio alle rotazioni:
    \begin{align*}
    M_1 (T_x, &T_y, M_z)=  M_1 (T_x, T_y) = \sum^m_{j=1}2q_j\Omega_{1j} \quad\quad\quad dove:\\
    \Omega_{11}&=0 \quad\quad\quad \Omega_{12}=0 \quad\quad\quad \Omega_{13}=ah \quad\quad\quad   \\
     \Omega_{14}&=\frac{ah}{2} \quad\quad\quad \Omega_{15}=\frac{ah}{2} \quad\quad\quad \Omega_{16}=0 \quad\quad\quad    \\
\end{align*}

\begin{equation*}
    \begin{cases}
        M_1(T_y)=M_z+Tya\\
    \sum^m_{j=1}2q_j\Omega_{1j} = -2(q_3ah+q_4\frac{ah}{2}+q_5\frac{ah}{2})
    \end{cases}
\end{equation*}

Chiudendo così un sistema di 6 equazioni e 6 incognite.\\
Si può anche tagliare un pannello ed esprimere i flussi come:
\begin{equation*}
    q_j\,=\,q'_j\,+\,   \sum^N_{k=1}\,\alpha_{jk}\,q^*_k
\end{equation*}
Questa tecnica consiste nel sovrapporre gli effetti dei $q'_j$ e dei $q^*_k$. I primi sono i flussi nel caso ogni cella fosse tagliata in un pannello e quindi aperta e labile rispetto alla torsione. I secondi sono i flussi che circolano in ogni cella per rispettare l'equilibrio alle rotazioni.


    
\end{esempioBox}


