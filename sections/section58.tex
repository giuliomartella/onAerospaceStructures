\section{Ingobbamento. Applicazione a casi semplici}

Partendo dall'equazione per l'ingobbamento:
\begin{align*}
     s_{2j}-s_{1j} \,&=\,\frac{q_j\,l_j}{Gt_j}- 2\Omega_{Oj}\dot{\theta}\\
\end{align*}

\begin{esempioBox}
    Consideriamo il caso di una semplice sezione a una cella rettangolare, di dimensioni $2a\times a$, ove è applicata solo una coppia $M_z$.
\begin{align*}
    q\,&=\, \frac{M_z}{2\Omega}\,=\,\frac{M_z}{4a^2}\\
     \dot{\theta}  \,&=\,\frac{1}{2G\Omega}\sum^m_{j=1}\frac{q_j\,l_j}{t_j}\\
     &=\,\frac{1}{2G\,2a^2}\frac{M_z}{4a^2t}(a+a+2a+2a)\\
     &=\,\frac{3}{8}\frac{M_z}{G\,a^3t}\\
\end{align*}
Prendo come polo il centro della sezione per cui:
\begin{align*}
    \Omega_1\,&=\,2a\,\frac{a}{2}\frac{1}{2}\,=\,\frac{a^2}{2}\,=\,\Omega_3\\
     \Omega_2\,&=\,a\,\frac{2a}{2}\frac{1}{2}\,=\,\frac{a^2}{2}\,=\,\Omega_4
\end{align*}

Per cui calcolo gli spostamenti relativi fra le coppie di correnti:
\begin{align*}
     s_{2}-s_{1} \,&=\,\frac{M_z}{4a^2}2a\frac{1}{Gt}-2\left( \frac{a^2}{2} \right) \left(   \frac{3}{8}\frac{M_z}{G\,a^3t}\right)
     = \frac{1}{8}\frac{M_z}{Gta} \,=\,\Delta s_M\\
      s_{3}-s_{2} \,&=\,\frac{M_z}{4a^2}a\frac{1}{Gt}-2\left( \frac{a^2}{2} \right) \left(   \frac{3}{8}\frac{M_z}{G\,a^3t}\right)
     = -\frac{1}{8}\frac{M_z}{Gta} \,=\,-\Delta s_M\\
      s_{4}-s_{3} \,&=\,\frac{M_z}{4a^2}2a\frac{1}{Gt}-2\left( \frac{a^2}{2} \right) \left(   \frac{3}{8}\frac{M_z}{G\,a^3t}\right)
     = \frac{1}{8}\frac{M_z}{Gta} \,=\,\Delta s_M\\
      s_{1}-s_{4} \,&=\,\frac{M_z}{4a^2}a\frac{1}{Gt}-2\left( \frac{a^2}{2} \right) \left(   \frac{3}{8}\frac{M_z}{G\,a^3t}\right)
     =- \frac{1}{8}\frac{M_z}{Gta} \,=\,-\Delta s_M\\
\end{align*}

Come riferimento impostiamo il primo corrente, per cui:
\begin{align*}
    s_1\,&=\,0\\
     s_2\,&=\,\Delta s_M\\
     s_3-\Delta s_M \,&=\,-\Delta s_M \qquad\rightarrow\qquad s_3\,=\,0\\
      s_4-0\,&=\,\Delta s_M \qquad\rightarrow\qquad s_4\,=\,\Delta s_M\\
\end{align*}

Gli spostamenti sono relativi quindi li redistribuisco per una migliore visualizzazione:
\begin{equation*}
    s_1\,=\,s_3\,=\,-\frac{\Delta s_M}{2}\qquad\qquad   s_2\,=\,s_4\,=\,\frac{\Delta s_M}{2}
\end{equation*}
    
\end{esempioBox}



\begin{esempioBox}
    Consideriamo la stessa sezione del caso precedente, questa volta imponendo solo un taglio $T_y$ al centro. Per simmetria i flussi sono immediati:
    \begin{align*}
     q_1 \,&=\, 0\\
      q_2 \,&=\,-\frac{T_y}{2a} \\
      q_3 \,&=\,0 \\
      q_4 \,&=\,\frac{T_y}{2a} \\
\end{align*}

Il taglio passa per il CT, la torsione è nulla. Calcolo gli spostamenti con la formula dell'ingobbamento.

\begin{align*}
     s_{2}-s_{1} \,&=\,0\\
      s_{3}-s_{2} \,&=\,-\frac{T_y}{2a} \frac{a}{Gt}\,=\,  -\frac{T_y}{2Gt}\,=\, -\Delta s_T\\
      s_{4}-s_{3} \,&=\,0\\
      s_{1}-s_{4} \,&=\,\frac{T_y}{2a} \frac{a}{Gt}\,=\,  \frac{T_y}{2Gt}\,=\, \Delta s_T
\end{align*}

Gli spostamenti sono relativi quindi li redistribuisco per una migliore visualizzazione:
\begin{equation*}
    s_1\,=\,s_2\,=\,\frac{\Delta s_T}{2}\qquad\qquad   s_3\,=\,s_4\,=\,-\frac{\Delta s_T}{2}
\end{equation*}
    
\end{esempioBox}


Si presenta quindi un problema di compatibilità tra la deformazione mostrata e il vincolo di estremità con una faccia piatta (all'incastro).\\
Questi sono raccordati da un ulteriore sistema di deformazione che rende la deformata totale congruente ai vincoli.\\
Considero al soluzione fondamentale che conosciamo e la sommo a una soluzione correttiva (ignota), ripristinando la congruenza. La soluzione totale diventa:
\begin{align*}
    \left\{ q\right\}_t \,&=\,   \left\{ q\right\}_f  +   \left\{ q\right\}_c \\
       \left\{ N\right\}_t \,&=\,   \left\{ N\right\}_f  +   \left\{ N\right\}_c \\
\end{align*}

Dato che  l'equilibrio rispetto ai carichi esterni è dato solo dalla soluzione fondamentale, la soluzione correttiva è autoequilibrante.\\
Non necessariamente la soluzione correttiva esiste. Ad esempio una sezione a quattro correnti sottoposta a taglio pura ha una deformata con faccia piana, quindi rispetta il vincolo di incastro rigido.


