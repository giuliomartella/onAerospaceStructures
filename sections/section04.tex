\section{Equazioni indefinite di equilibrio}

Per scrivere le equazioni indefinite di equilibrio consideriamo un parallelepipedo infinitesimo nello spazio, con un vertice nell'origine degli assi e i lati che si propagano delle quantità $dx$, $dy$ e $dz$. Le sue facce sono ortogonali agli assi cartesiani.

\subsection*{Equilibrio nelle tre direzioni}
Consideriamo lo sviluppo in serie di Taylor per l'andamento degli sforzi lungo le tre direzioni, per cui risulta: 

\begin{equation*}
\sigma_{ij}(x_i + \mathrm{d}x_i) = \sigma_{ij}(x_i) + \frac{\partial \sigma_{ij}}{\partial x_i} \, \mathrm{d}x_i
\end{equation*}

Scriviamo ora, ad esempio per la direzione $x$, il bilancio delle forze considerando lo sviluppo di Taylor degli sforzi sulle facce opposte. $b_x$ è la componente di forza di volume. Si assume una convenzione di segno positiva delle trazioni uscente dalla faccia con normale positiva.

\begin{align*}
&(\sigma_{xx} + \frac{\partial \sigma_{xx}}{\partial x} dx) dy\, dz - \sigma_{xx} dy\, dz \\
+ &(\sigma_{yx} + \frac{\partial \sigma_{yx}}{\partial y} dy) dx\, dz - \sigma_{yx} dx\, dz \\
+ &(\sigma_{zx} + \frac{\partial \sigma_{zx}}{\partial z} dz) dx\, dy - \sigma_{zx} dx\, dy \\
+ &b_x dx\, dy\, dz = 0
\end{align*}

Da cui risulta, alleggerendo la notazione:
\begin{equation*}
\sigma_{xx/x}  + \sigma_{yx/y}  + \sigma_{zx/z}  + b_x = 0
\end{equation*}

Analogamente il bilancio delle forze sulle direzioni $y$ e $z$:
\begin{equation*}
\sigma_{xy/x}  + \sigma_{yy/y}  + \sigma_{zy/z}  + b_y = 0
\end{equation*}

\begin{equation*}
\sigma_{xz/x}  + \sigma_{yz/y}  + \sigma_{zz/z}  + b_z = 0
\end{equation*}

Per le tre direzioni quindi vale (in diverse notazioni):
\begin{equation*}
\sigma_{ij/i}    + b_i = 0   \quad \quad  \mathrm{div}\, \boldsymbol{\sigma} + \mathbf{b} = \mathbf{0}
\end{equation*}

\subsection*{Equilibrio alle rotazioni}

Consideriamo l'equilibrio dei momenti dovuti alle forze di taglio sulle facce laterali attorno all'origine del sistema, rispetto a un asse parallelo ad $x$.

\begin{align*}
&(\sigma_{yz} + \sigma_{yz/y} dy) \ dx dz\, \frac{dy}{2} + \sigma_{yz}\ dx dz\, \frac{dy}{2}  \\
- \ &(\sigma_{zy} + \sigma_{zy/z} dz)\ dx dy\, \frac{dz}{2} - \sigma_{zy}\ dx dy\, \frac{dz}{2}  \\
  =& \ 0
\end{align*}

Trascurando gli infinitesimi di ordine superiore risulta $\sigma_{yz} = \sigma_{zy}$, analogamente:
\begin{equation*}
\sigma_{xy} = \sigma_{yx}   \quad\quad \sigma_{xz} = \sigma_{zx}
\end{equation*}

Quindi, se non sono applicati momenti esterni centrati sul volume infinitesimo, $\boldsymbol{\sigma}$ è simmetrico, in tre dimensioni è quindi definito da 6 gradi di libertà (da tre componenti normali e tre componenti tangenziali indipendenti).