\section{Materiali ortotropi}

Un materiale ortotropo è un materiale le cui proprietà, e quindi la risposta alle sollecitazioni, presentano tre piani di simmetria mutuamente perpendicolari.\\
In particolare, i materiali compositi sono generalmente materiali bifase, composti da fibre e matrice.\\
Le fibre hanno forma di fili sottili e presentano alta rigidezza e resistenza, ma non hanno forma propria. La matrice ha bassa rigidezza e resistenza ma ha il compito di mantenere la forma e trasferire il carico alle fibre.

Le proprietà meccaniche delle fibre sono molto superiori di quelle dei metalli e anche quando unite con la matrice rimangono superiori a quelle dei metalli.\\
I materiali compositi presentano problematiche su giunzioni, fatica, impatti e difficoltà di progettazione.

Studiamo il materiale partendo dalla singola lamina, dove le fibre hanno orientazione $\theta$. Il sistema di riferimento originale è $x, y$, quello rispetto all'ortotropia, quindi ruotato di   $\theta$, è $n, t$.

Per studiare la risposta del materiale:
\begin{enumerate}
    \item Lo carico nella direzione dell'ortotropia. Sollecitando lungo queste direzioni la risposta è simmetrica, senza accoppiamento tra sforzi e deformazioni ($\frac{\nu_{12}}{E_{22}}=\frac{\nu_{21}}{E_{11}}$).
    \begin{equation*}
\left\{
\begin{array}{c}
\varepsilon_{1} \\
\varepsilon_{2} \\
\gamma_{12}
\end{array}
\right\}
=
\left[
\begin{array}{ccc}
\displaystyle \frac{1}{E_{11}} & \displaystyle-\frac{\nu_{12}}{E_{22}} & 0\\
\displaystyle -\frac{\nu_{21}}{E_{11}} & \displaystyle\frac{1}{E_{22}} &0 \\
0& 0 & \displaystyle\frac{1}{G}
\end{array}
\right]\,
\left\{
\begin{array}{c}
\sigma_{1} \\
\sigma_{2} \\
\tau_{12}
\end{array}
\right\}
\end{equation*}
    \\
    \item Se carico in una direzione diversa nasce una deformazione con l'accoppiamento di allungamento e taglio, nascono degli scorrimenti.
\end{enumerate}

Il materiale è caratterizzato da 4 parametri:  $E_{11}$,$E_{22}$,  $\nu_{12}$,  $G$.

Le lamine possono essere fatte anche come tessuti, avendo le fibre disposte nelle due direzioni. Si osserva che nel caso dei tessuti si ottiene una $E$ inferiore alla metà di quella dell'unidirezionale in quanto le fibre devono passare le une sopra le altre, piegandosi anche in condizione indeformata. Caricando la fibra a trazione lei si può stendere, riducendo la $E$.

\subsubsection*{Contesto industriale dei materiali compositi}
L'uso dei materiali compositi nell'industria, specie in quella aeronautica, ha subito un ritardo importante rispetto alle previsioni di crescita, possiamo elencare i principali motivi.

\begin{enumerate}
    \item Giunzioni. Per ottimizzare la struttura vanno prodotte parti grandi e continue così da non interrompere le fibre. Le giunzioni con chiodatura per esempio danneggiano le fibre e causano delaminazione, le bullonature vengono usate solo dove è fondamentale lo smontaggio, anche se in queste eccezioni si rinforza la struttura con parti in metallo. Generalmente viene usata la colla o la stessa resina del materiale, tuttavia gli incollaggi agiscono solo sulla matrice.\\
    \item Fatica. Mentre i metalli soffrono di fatica a trazione i compositi ne soffrono a compressione, questo aspetto rende necessaria una progettazione diversa.\\
    \item Impatti. Il carbonio è fragile. Piccoli impatti possono causare danni interni non visibili ma col tempo portare alla delaminazione. Si usano protezioni con vernici speciali o fibre di kevlar. Inoltre è difficile fare indagini non distruttive sulla struttura.\\
    \item Progettazione. I materiali ortotropi rendono non validi i modelli della meccanica strutturale precedenti. Le deformazioni sono in direzioni non allineate al carico. Questo punto introduce una difficoltà di progettazione ma al tempo stesso una possibilità.\\
    \item Sostenibilità. Mentre il metallo è facilmente recuperabile e riutilizzabile, i materiali compositi sono molto difficili e costosi da smaltire al termine della loro vita utile.
\end{enumerate}

