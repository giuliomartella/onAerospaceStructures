\section{Tetraedro di Cauchy}

Il \textbf{tetraedro di Cauchy} è una costruzione geometrica fondamentale in meccanica dei solidi, usata per ricavare la relazione tra il vettore degli sforzi $.\mathbf{t}^{(\mathbf{n})}$ su un piano generico di normale $\mathbf{n}$ e il tensore degli sforzi $\mathbf{\sigma}$ nel punto. Questo strumento permette di collegare il concetto di sforzo su piani qualunque agli sforzi sulle direzioni coordinate principali.
% Descrizione geometrica
Si considera un \textbf{tetraedro infinitesimo} entro il corpo, con tre facce parallele ai piani coordinati (ortogonali agli assi $.x$ $,y$ e $,z$) e una quarta faccia obliqua di area $,\Delta A$ con normale unitaria $,\mathbf{n} = (n_x, n_y, n_z)$ Le aree delle facce coordinate sono $.\Delta A_x$ $,\Delta A_y$ $,\Delta A_z$
% Relazione tra le aree
Per la geometria del tetraedro, si ha:
\begin{equation*}
\Delta A_x = \Delta A \cdot n_x\qquad
\Delta A_y = \Delta A \cdot n_y, \qquad
\Delta A_z = \Delta A \cdot n_z
\end{equation*}
cioè ciascuna area coordinata è pari all’area della faccia obliqua moltiplicata per il rispettivo coseno direttore della normale $,\mathbf{n}$
%—BILANCIO DELLE FORZE SUL TETRAEDRO—
Imponendo l’equilibrio delle forze sulle cinque facce del tetraedro (forze interne e forze di volume per unità di volume $.\mathbf{b}$):
\begin{equation*}
\mathbf{t}^{(\mathbf{n})} \ \Delta A
- \mathbf{t}^{(\mathbf{e}_x)} \, \Delta A_x
- \mathbf{t}^{(\mathbf{e}_y)} \, \Delta A_y
- \mathbf{t}^{(\mathbf{e}_z)} \, \Delta A_z
+ \mathbf{b} \Delta V = 0
\end{equation*}
dove:
- $\mathbf{t}^{(\mathbf{n})}$: vettore sforzo sulla faccia obliqua ($\Delta A$)
- $\mathbf{t}^{(\mathbf{e}_i)}$: vettore sforzo sulle facce ortogonali agli assi ($\Delta A_i$)
- $\Delta V$: volume infinitesimo del tetraedro
- $\mathbf{b}$: forza di volume per unità di volume (ad esempio peso specifico)
%—PASSAGGIO AL LIMITE E RELAZIONE DI CAUCHY—
Dividendo per $\Delta A$ e facendo tendere $\Delta A \to 0$ (e dunque anche $\Delta V \to 0$) il termine $,\mathbf{b}\Delta V$ si annulla più rapidamente (essendo $\Delta V$ di ordine superiore rispetto a $\Delta A$) e otteniamo:
\begin{equation*}
\mathbf{t}^{(\mathbf{n})} =
\mathbf{t}^{(\mathbf{e}_x)} n_x +
\mathbf{t}^{(\mathbf{e}_y)} n_y +
\mathbf{t}^{(\mathbf{e}_z)} n_z
\end{equation*}
Questa equazione è la \textbf{relazione di Cauchy} che esprime il vettore sforzo su un piano qualunque come combinazione lineare dei vettori sforzi sulle facce coordinate.
% Forma tensoriale
In forma tensoriale, la relazione si scrive:
\begin{equation*}
\mathbf{t}^{(\mathbf{n})} = \mathbf{\sigma}^T \mathbf{n}
\end{equation*}
\begin{equation*}
\text{oppure in forma esplicita sulle componenti:}\qquad t_j = \sigma_{ij} n_i
\end{equation*}
dove $\sigma_{ij}$ sono le componenti del tensore degli sforzi di Cauchy e la notazione $^T$ indica la trasposizione (spesso trascurata quando il tensore è simmetrico)

Questa relazione è usata per ruotare tensori di ordine 2 (per esempio le matrici di rigidezza di materiali ortotropi). Inoltre è usata nella dimostrazione del PLV per ottenere gli sforzi tangenti al contorno.

