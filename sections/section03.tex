
\section{Tetraedro di Cauchy}
Consideriamo un punto materiale $P$ all’interno di un corpo continuo. Attorno a $P$ costruiamo un \textbf{tetraedro infinitesimo} le cui quattro facce siano:
\begin{itemize}
    \item una faccia obliqua, la cui normale unitaria sia $\mathbf{n} = (n_x, n_y, n_z)$;
    \item tre facce coordinate, perpendicolari rispettivamente agli assi $x$, $y$ e $z$.
\end{itemize}

Sulla faccia obliqua agisce il \textbf{vettore sforzo} $\mathbf{t}^{(\mathbf{n})}$ (o \textit{vettore delle trazioni}) relativo alla normale $\mathbf{n}$.  
Sulle tre facce coordinate agiscono invece i vettori sforzo $\mathbf{t}^{(x)}$, $\mathbf{t}^{(y)}$ e $\mathbf{t}^{(z)}$, diretti secondo le forze che il materiale circostante esercita sulla faccia considerata.

\subsubsection*{Geometria del tetraedro}
Il tetraedro è costruito in modo che il vertice opposto alla faccia obliqua coincida con il punto $P$, e le tre facce coordinate siano contenute nei piani $x = 0$, $y = 0$ e $z = 0$.  
Se $\Delta S_n$ è l’area della faccia obliqua, la proiezione di quest’area sulle facce coordinate fornisce:
\begin{equation*}
\Delta S_x = n_x \, \Delta S_n, \quad
\Delta S_y = n_y \, \Delta S_n, \quad
\Delta S_z = n_z \, \Delta S_n
\end{equation*}
dove $n_x$, $n_y$, $n_z$ sono le componenti della normale unitaria $\mathbf{n}$.

\subsubsection*{Equilibrio del tetraedro}
Poiché il tetraedro è infinitesimo, le forze di volume sono proporzionali al volume ($O(\Delta V)$) mentre le forze superficiali sono proporzionali all’area ($O(\Delta S)$). Nel limite $\Delta V \to 0$, le forze di volume diventano trascurabili rispetto alle forze superficiali, e l’equilibrio statico impone:
\begin{equation*}
\mathbf{t}^{(\mathbf{n})} \, \Delta S_n + \mathbf{t}^{(x)} \, \Delta S_x + \mathbf{t}^{(y)} \, \Delta S_y + \mathbf{t}^{(z)} \, \Delta S_z = \mathbf{0}
\end{equation*}

Sostituendo le relazioni geometriche e dividendo per $\Delta S_n$ si ottiene:
\begin{equation*}
\mathbf{t}^{(\mathbf{n})} + n_x \, \mathbf{t}^{(x)} + n_y \, \mathbf{t}^{(y)} + n_z \, \mathbf{t}^{(z)} = \mathbf{0}
\end{equation*}

\subsubsection*{Forma tensoriale}
Introduciamo il \textbf{tensore degli sforzi di Cauchy} $\boldsymbol{\sigma}$, le cui colonne sono i vettori sforzo sulle facce coordinate:
\begin{equation*}
\boldsymbol{\sigma} =
\begin{bmatrix}
| & | & | \\
\mathbf{t}^{(x)} & \mathbf{t}^{(y)} & \mathbf{t}^{(z)} \\
| & | & |
\end{bmatrix}
\end{equation*}

La relazione di equilibrio si riscrive come:
\begin{equation*}
\mathbf{t}^{(\mathbf{n})} = \boldsymbol{\sigma} \, \mathbf{n}
\end{equation*}
Questa è la \textbf{legge di Cauchy}, che permette di calcolare il vettore sforzo agente su una qualsiasi faccia conoscendo il tensore degli sforzi e la normale alla faccia stessa.
