\section{Modello di trave. Definizione di centro di taglio e differenze nell'uso degli assi di riferimento.}

Il centro di taglio (CT) è il punto per il quale, applicando delle forze di taglio $T_x$ e $T_y$, l'energia di deformazione associata alle $\tau$ è disaccoppiata:

\begin{equation*}
    V_{d}= \frac{1}{2} \int_l 
    \left[
        \frac{T_x^2}{G A^*_x} 
        + \frac{T_y^2}{G A^*_y} 
        + \frac{M_z^2}{G J_t}
    \right] dz
\end{equation*}

Riguardo la scelta del sistema di riferimento, per calcolare $T_z$, $M_x$, $M_y$ uso il sistema principale d'inerzia, per  $T_x$, $T_y$, $M_z$ uso il sistema, con il centro nel centro di taglio e gli assi orientati come quelli del sistema principale d'inerzia. Per verificare questa affermazione chiamiamo questo modo di procedere $A$ e lo confrontiamo con $B$ dove calcoliamo tutte le quantità rispetto al baricentro e $C$, dove calcoliamo tutte le quantità rispetto al centro di taglio. Ipotizzo CG e CT allineati sull'asse $z$, la distanza fra loro vale $d$.\\
Poiché i tre sistemi sono equilibranti rispetto ai carichi sono anche equivalenti fra loro.
\begin{itemize}
    \item Equivalenza fra $A$ e $B$:
    \begin{align*}
        T_x^B &= T_x^A \quad\quad\quad \quad &M_x^B &= M_x^A \\
        T_y^B &= T_y^A \quad\quad\quad \quad &M_y^B + T_z^Bd &= M_x^A+T_z^Ad \\
        T_z^B &= T_z^A \quad\quad\quad \quad &M_x^B -T_y^Bd&= M_x^A \\
    \end{align*}
    ritrovo quindi un errore nel calcolo della torsione, una componente importante nella trattazione delle ali.\\
    \item Equivalenza fra $A$ e $C$:
    \begin{align*}
        T_x^C &= T_x^A \quad\quad\quad \quad &M_x^C &= M_x^A \\
        T_y^C &= T_y^A \quad\quad\quad \quad &M_y^C - T_z^Cd &= M_x^A \\
        T_z^C &= T_z^A \quad\quad\quad \quad &M_x^C &= M_x^A \\
    \end{align*}
    in questo caso sbaglio il calcolo della flessione.\\    
\end{itemize}
Se $T_z$ è piccola o nulla l'errore commesso scegliendo l'opzione $C$ è piccolo. Nelle ali non ci sono azioni dirette come l'asse $z$, dunque l'errore commesso è trascurabile. Questa considerazione può non valere per le pale degli elicotteri, sottoposte a una forte azione assiale per via della forza centrifuga, per queste usiamo un modello di trave a due assi.



