\section{Laminazioni particolari per i materiali compositi}

\begin{itemize}
    \item Laminazione simmetrica:
    Se ho un angolo $+\theta$ a quota $+z$ ho anche $+\theta$ a quota $-z$ (e le due lamine a queste quote sono fatte dello stesso materiale).
    In questo modo ottengo $[B]=0$, elimino i legami fra le forze nel piano e flessioni/torsioni (rimangono i legami assiale-taglio e flessione-torsione).\\
    \item Laminazione equilibrata:
    A ogni lamina con un angolo  $+\theta$ ne corrisponde un'altra con angolo  $-\theta$ (dello stesso materiale).
    Ottengo $A_{13}=A_{23}=A_{31}=A_{32}=0$, eliminando il legame fra forze assiali e forze di taglio.\\
    \item Laminazione bilanciata:
    Se ho un angolo $+\theta$ a quota $+z$ ho anche $-\theta$ a quota $-z$ (e le due lamine a queste quote sono fatte dello stesso materiale).
    Viene sottintesa la laminazione equilibrata, inoltre  $C_{13}=C_{23}=C_{31}=C_{32}=0$, elimino l'accoppiamento flessione-torsione
\end{itemize}

Con una laminazione simmetrica e bilanciata ho $[B]=0$ e disaccoppiamenti (assiale-taglio, flessione-torsione) ottenendo un materiale quasi isotropo. Questo è ottenibile con i tessuti.\\
In un materiale isotropo $E$,$\nu$,$G$ sono legati e solo due di loro sono indipendenti. In un materiale quasi isotropo ho un comportamento simile senza gli stessi legami fra le grandezze.

Il libro di laminazione consiste nell'affiancare più pagine, ognuna con angoli diversi. \\
Per ogni lamina conosco $[\bar{D}]_i$ nel sistema di riferimento dell'ortotropia e conosco $[T]_i$.\\
Ruoto tutte le $[\bar{D}]_i$ per portarle nel sistema di riferimento del laminato:
\begin{equation*}
    [D]_i =  [T]_i [\bar{D}]_i[T]_i^T
\end{equation*}
A partire dalle $[D]_i$ posso ricavare $[A]_i$, $[B]_i$, $[C]_i$.

Posso scrivere:
\begin{equation*}
\left\{
\begin{array}{c}
N\\M
\end{array}
\right\}
=
\begin{bmatrix}
[A] & [B] \\
[B] & [C]
\end{bmatrix}
\left\{
\begin{array}{c}
\boldsymbol{\varepsilon_0}\\
\boldsymbol{k}
\end{array}
\right\}
\end{equation*}
E ricavare:
\begin{equation*}
\left\{
\begin{array}{c}
\boldsymbol{\varepsilon_0}\\
\boldsymbol{k}
\end{array}
\right\}= \begin{bmatrix}
[A] & [B] \\
[B] & [C]
\end{bmatrix}^{-1}
\left\{
\begin{array}{c}
N\\M
\end{array}
\right\}
\end{equation*}


Per ogni lamina calcolo:
\begin{align*}
     \left\{\varepsilon\right\}_{\text{lower}_i} &=    \left\{\varepsilon_0\right\}_i+     h_{i-1}  \left\{k\right\}_i\\
       \left\{\varepsilon\right\}_{\text{upper}_i} &=    \left\{\varepsilon_0\right\}_i+     h_{i}  \left\{k\right\}_i
\end{align*}
Ovvero le deformazioni sui bordi inferiore e superiore della lamina, differenti a causa della flessione. Nel sistema di riferimento del laminato scrivo:

\begin{align*}
     \left\{\sigma\right\}_{l_i} &= [D]_i \left\{\varepsilon\right\}_{l_i}\\
     \left\{\sigma\right\}_{u_i} &= [D]_i \left\{\varepsilon\right\}_{u_i}\\
\end{align*}

Per analizzarle le porto nel sistema di riferimento della lamina, ovvero dell'ortotropia:
\begin{align*}
     \left\{\bar{\sigma}\right\}_{l_i} &= [T]^{-1}  \left\{\sigma\right\}_{l_i}\\
     \left\{\bar{\sigma}\right\}_{u_i} &= [T]^{-1}   \left\{\sigma\right\}_{u_i}
\end{align*}

Posso quindi verificare che $\bar{\sigma}_i$ siano ammissibili attraverso il criterio più adatto.
