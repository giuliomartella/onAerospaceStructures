\section{Lavoro di deformazione ed energia di deformazione per una sezione a semiguscio.}

Energia di deformazione associata alle $\tau$, assumendo il modello di trave:
\begin{equation*}
    \delta V_d \,=\, \frac{1}{2}\int_\Omega \tau\,\gamma \, dV \,=\, \int_L\int_{\bar{A}} (\gamma_x\tau_x+ \gamma_y\tau_y) \, dA\,dz
\end{equation*}

Considerando ora che i correnti non hanno sforzi $\tau$, che sono invece solo nei pannelli, scriviamo:

\begin{align*}
    V_{d\tau} &= \sum^m_{j=1} V_{d\tau_j} \\
    q_j&=\tau_j\,t_j    \quad \quad \quad\rightarrow \quad \quad \quad \tau_{\mathrm{media}_j}=\frac{q_j}{t_j}\\
    \gamma_j &=\frac{\tau_j}{G} = \frac{q_j}{Gt_j}\\
\end{align*}

Quindi l'energia di deformazione legata alle $\tau$ per ogni singolo pannello è:
\begin{align*}
    V_{d\tau_j} \,&=\, \frac{1}{2}\int_{\Omega_j} \tau_j\,\gamma_j \, dV \,\\
    &=\,\frac{1}{2} \int_L\int_{\bar{A}_j}\tau_j\,\gamma_j \, dA\,dz\\
     &=\,\frac{1}{2} \int_L\int_{l_j}\int_{t_j}  \left( \frac{q_j}{t_j} \,\frac{q_j}{Gt_j} \right)     \, dt\,dl\,dz\\
     &=\, \frac{1}{2} \int_L \,\frac{q_j^2}{Gt_j^2}\left( t_j\,l_j \right) \,dz\\
         &=\, \frac{1}{2} \int_L \,\frac{q_j^2}{Gt_j}\,l_j \,dz
\end{align*}

Per tutti i pannelli:
\begin{equation*}
     V_{d\tau} \,=\, \frac{1}{2} \int_L \,\sum^m_{j=1}\frac{q_j^2}{Gt_j}\,l_j \,dz
\end{equation*}

Il lavoro di deformazione legato alle $\tau$ per ogni pannello è:

\begin{equation*}
     \delta L_{d\tau} \,=\,  \int_L \,\sum^m_{j=1}\frac{q_j\delta q_j}{Gt_j}\,l_j \,dz
\end{equation*}

Se uso il centro di taglio per calcolare le azioni interne, rispetto all'energia di deformazione, vale:

\begin{equation*}
    V_{d\tau}= \frac{1}{2} \int_l 
    \left[
        \frac{T_x^2}{G A^*_x} 
        + \frac{T_y^2}{G A^*_y} 
        + \frac{M_z^2}{G J_t}
    \right] dz
\end{equation*}

Partendo da una struttura, applicando uno alla volta $T_x$, $T_y$ ed $M_z$ posso calcolare rispettivamente $A^*_x$, $A^*_y$ e $J_y$.

Possiamo confrontare l'espressione per il calcolo dell'energia di deformazione associata alle $\tau$ con l'energia associata alle $\sigma$:

\begin{equation*}
    V_{d\sigma}= \frac{1}{2} \int_l 
    \left[
        \frac{T_z^2}{E\bar{A}} 
        + \frac{M_x^2}{E J_x}
        + \frac{M_y^2}{E J_y}
    \right] dz
\end{equation*}
Notiamo come per una trave allungata $V_{d\sigma}>>V_{d\tau}$, il contrario per una trave tozza.
