\section{Modello di trave. Legame spostamenti, deformazioni e azioni interne.}

Per un generico punto della sezione vale:
 \begin{equation*}
        s_z \,=\,  w_z \,-\,w'_x  x\, -\,w'_y  y
 \end{equation*}

 Considero come componenti significative di $\left\{\sigma\right\}$ e $\left\{\varepsilon\right\}$ :
 \begin{equation*}
     \left\{\sigma\right\} = \left\{
    \begin{array}{c}
    \sigma_{zz} \\ \sigma_{zx} \\ \sigma_{zy}
    \end{array}
    \right\} = \left\{
    \begin{array}{c}
    \sigma\\ \tau_x \\ \tau_y
    \end{array}
    \right\}\quad\quad\quad\quad
    \left\{\varepsilon\right\} = \left\{
    \begin{array}{c}
    \varepsilon_{zz} \\ \gamma_{zx} \\ \gamma_{zy}
    \end{array}
    \right\} = \left\{
    \begin{array}{c}
    \varepsilon\\ \gamma_x \\ \gamma_y
    \end{array}
    \right\}
 \end{equation*}

 Dalle precedenti vale:
 \begin{align*}
     \varepsilon_{zz} &= \frac{1}{2} (s_{z/z} +s_{z/z})= s_{z/z} =  w'_z \,-\,w''_x  x\, -\,w''_y  y\\
     \sigma_{zz} &= \sigma = E \,\varepsilon = E\,w'_z \,-\,E\,w''_x  x\, -\,E\,w''_y  y
 \end{align*}

 Scrivo quindi le equazioni di equivalenza per le azioni interne:
 \begin{align*}
    T_z \, &=\, \int_{\bar{A}} \sigma \, dA = \, \int_{\bar{A}} (E\,w'_z \,-\,E\,w''_x  x\, -\,E\,w''_y  y) \, dA \\
    M_y \, &=\, - \int_{\bar{A}} (E\,w'_zx \,-\,E\,w''_x  x^2\, -\,E\,w''_y  xy)  \, dA \\
    M_x \, &=\,  \int_{\bar{A}} (E\,w'_z y\,-\,E\,w''_x  xy\, -\,E\,w''_y  y^2) \, dA \\
\end{align*}

Se la trave è omogenea $E$ esce dagli integrali mentre se il sistema è principale d'inerzia i momenti statici e il momento centrifugo si annullano. Risultando
\begin{equation*}
    \begin{cases}
        T_z = E\bar{A} w'_z\\
        M_x = -EJ_xw''_y\\
        M_y = EJ_yw''_x
    \end{cases}
\end{equation*}
Estendendo al lavoro virtuale di deformazione (PLV), considerando solo $\sigma_{zz}$:
\begin{align*}
     L_d &= \int_\Omega  \left\{\varepsilon\right\}^T \left\{\sigma\right\}\, dV\\
     &= \int_l\int_{\bar{A}} ( \delta w'_z \,-\,\delta w''_x  x\, -\,\delta w''_y  y)   (E\,w'_z \,-\,E\,w''_x  x\, -\,E\,w''_y  y)\,dAdz\\
     &= \int_l(\delta w'_z w'_z E\bar{A}\,-\,\delta w''_xw''_x EJ_y  \, -\,\delta w''_yw''_yEJ_x  )\,dz
\end{align*}
Ritrovandoci nel metodo degli spostamenti questa è un'equazione del PLV, la uso per scrivere condizioni di equilibrio.



