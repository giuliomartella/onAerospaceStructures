\section{Incollaggi. Andamento degli sforzi e dei flussi.}


Attraverso un incollaggio il flusso di taglio è trasferito in modo continuo e graduale. Il flusso, mentre sta attraversando l'incollaggio, può essere considerato come la somma di due funzioni $q_1(x)+q_2(x)$ per cui, su un incollaggio lungo $2b$:
\begin{align*}
    q_1(b) &= q\qquad &q_1(-b)&=0\\
     q_2(b) &= 0\qquad &q_2(-b)&=q\\
\end{align*}

Verifichiamo con l'equazione dei flussi sorgenti che $q_1+q_2=q$.

Seziono l'incollaggio per un $dx$ piccolo e analizzo gli sforzi per le partizioni di pannello superiore, colla e pannello inferiore.
Scrivo l'equilibrio sul pannello inferiore in direzione $z$:
\begin{align*}
   & q_1\,dz - (q_1+dq_1)\,dz+\tau_c\,dx\,dz\,=\,0\\
    &dq_1\,=\,\tau_c\,dx\\
    &\tau_c\,=\, \frac{dq_1}{dx}\,=\,\dot{q_1}
\end{align*}

Per il pannello superiore:
\begin{align*}
   & q_2\,dz - (q_2+dq_2)\,dz-\tau_c\,dx\,dz\,=\,0\\
    &dq_2\,=\,-\tau_c\,dx\\
    &\tau_c\,=-\, \frac{dq_2}{dx}\,=\,-\dot{q_2}
\end{align*}
Queste condizioni, insieme a  $q_1+q_2=q$, rappresentano le uniche tre condizioni di equilibrio.
Consideriamo l'energia di deformazione per la colla come:
\begin{equation*}
    \tau_c\,\gamma_c\,dV\,=\,\tau_c\,\frac{\tau_c}{G_c}\,dy\,dx\,=\,\frac{\tau_c^2}{G_c} t \,dx
\end{equation*}
Introduciamo il coefficiente:
\begin{equation*}
    \frac{2}{\alpha^2}\,=\,\frac{t_c}{G_c}Gt
\end{equation*}
Scriviamo adesso il teorema di Menabrea per avere delle condizioni di congruenza.


\begin{align*}
    V_{d}\,&=\,V_d(q) +\frac{1}{2} \int_{-b}^b\left( \frac{q_1^2}{Gt}  + \frac{q_2^2}{Gt} + \frac{\tau_c^2}{G_c}\,t  \right)dx\\
   &=\,V_d(q) +\frac{1}{2} \int_{-b}^b\left( q_1^2+ (q-q_1)^2+ \frac{2}{\alpha^2} \dot{q_1}^2   \right)dx\, \frac{1}{Gt}\\
     &=\,V_d(q) +\frac{1}{2} \int_{-b}^b  F(q_1, \dot{q_1})\, dx\, \frac{1}{Gt}\\
\end{align*}

Per trovare il minimo dell'energia uso la minimizzazione del funzionale, per cui:
\begin{equation*}
    \frac{\partial F}{\partial q_1} - \left(  \frac{\partial F}{\partial \dot{q_1}} \right)'\,=\,0
\end{equation*}

\begin{equation*}
    \begin{cases}
          \frac{\partial F}{\partial q_1} = 2q_1-2(q-q_1)=4q_1-2q\\
           \frac{\partial F}{\partial \dot{q_1}}= \frac{4}{\alpha^2} \ddot{q_1} 
    \end{cases}
    \qquad \rightarrow\qquad
    -\frac{1}{\alpha^2} \ddot{q_1} +q_1= \frac{q}{2}
\end{equation*}

Per risolvere l'equazione differenziale ordinaria sommiamo la soluzione particolare e la soluzione dell'omogenea associata, per cui:
\begin{equation*}
    q_1\,=\,Ae^{\alpha x} +Be^{-\alpha x}+\frac{q}{2}
\end{equation*}
Per determinare le costanti $A$ e $B$ combiniamo le condizioni al contorno di $q_1$:
\begin{equation*}
    \begin{cases}
        Ae^{\alpha b} +Be^{-\alpha b}+\frac{q}{2}=q\\
        Ae^{-\alpha b} +Be^{\alpha b}+\frac{q}{2}=0\\
    \end{cases}
\end{equation*}

Sapendo che $2sh(x) = e^x-e^{-x}$ e che $sh(-x)=-sh(x)$, risulta:
\begin{equation*}
    A=-B \qquad A=-\frac{q}{2}\,\frac{1}{2\,sh(\alpha b)}
\end{equation*}
\begin{align*}
    q_1\,&=\,\frac{q}{2} \left(  1+\frac{sh(\alpha x)}{sh(\alpha b)}  \right)\\
     q_1\,&=\,\frac{q}{2} \left(  1-\frac{sh(\alpha x)}{sh(\alpha b)}  \right)\\
     \tau_c\,&=\, \dot{q_1}\,=\,\frac{q}{2} \alpha\left( \frac{ch(\alpha x)}{sh(\alpha b)}  \right)\\
\end{align*}

Il diagramma dei flussi corrispondente è simmetrico fra $q_1$ e $q_2$. Il diagramma degli sforzi di taglio teorico è asintotico agli estremi dell'incollaggio. Il diagramma reale ha comunque i picchi nella parte esterna.

Il creep è la deformazione progressiva del materiale in presenza di un carico costante e con valori di sforzo inferiori agli sforzi limiti di rottura.\\
La $\tau$ al centro dell'incollaggio è molto importante pur trasmettendo pochissima forza, infatti gli estremi lavorano al di sopra della $\tau^{creep}$, dunque dovrebbero scorrere via, sono tenuti invece dagli sforzi centrali $\tau<\tau^{creep}$ che agiscono da vincolo.
