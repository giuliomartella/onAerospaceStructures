\section{Dimostrazione dell'espressione del PLV}

Partiamo dalle equazioni indefinite di equilibrio $\mathrm{div}\,\boldsymbol{\sigma} + \mathbf{b} = \mathbf{0}$ e dalla relazione di Cauchy $\mathbf{t}_n = \boldsymbol{\sigma}^T \mathbf{n}$. Applichiamo le precedenti ad un corpo generico, con una parte del contorno vincolata (ovvero con spostamenti imposti) e con un'altra parte libera (con forze applicate note).

\begin{itemize}
    \item $S_u$: superficie vincolata 
    \begin{compactitem}
        \item $\mathbf{u}$ noti
        \item $\mathbf{t}_n$ incogniti
    \end{compactitem}
    
    \item $S_{\sigma}$: superficie libera 
    \begin{compactitem}
        \item $\mathbf{u}$ incogniti
        \item $\mathbf{t}_n$ noti (carichi applicati)
    \end{compactitem}
\end{itemize}

Scrivo i possibili lavori virtuali combinando il sistema di forze reali con gli spostamenti virtuali.
\begin{equation*}
    \delta \mathbf{u}^T (\mathrm{div}\,\boldsymbol{\sigma} + \mathbf{b}) = \mathbf{0}
\end{equation*}
Integro su tutto il volume e passo alla notazione di Einstein.
\begin{equation*}
    \int_{\Omega} \delta \mathbf{u}^T \left( \mathrm{div}\,\boldsymbol{\sigma} + \mathbf{b} \right) \, dV = \mathbf{0}
    \quad\quad
    \int_{\Omega} \delta u_j \left( \sigma_{ij/i} + b_j \right) \, dV = 0
\end{equation*}

Sommo e sottraggo $\int_{\Omega} \delta u_{j/i}  \sigma_{ij}  \, dV$.
\begin{equation*}
    \int_{\Omega}  \left(\delta u_j \sigma_{ij}  \right)_i \, dV  \, -\,\int_{\Omega} \delta u_{j/i}  \sigma_{ij}  \, dV \, + \, 
    \int_{\Omega} \delta u_{j}  b_j  \, dV= 0
\end{equation*}

Applico il teorema della divergenza al primo termine, assumendo che il dominio sia semplicemente connesso e che sforzi e spostamenti virtuali siano sufficientemente regolari.
\begin{equation*}
    \int_{S}  \delta u_j \sigma_{ij} N_i \, dA  \, -\,\int_{\Omega} \delta u_{j/i}  \sigma_{ij}  \, dV \, + \, 
    \int_{\Omega} \delta u_{j}  b_j  \, dV= 0
\end{equation*}

L'integrale di contorno si valuta sull'intera superficie, composta dalla parte vincolata e dalla parte libera. La prima però impone un vincolo che deve essere congruente con gli spostamenti virtuali.
\begin{align*}
    \int_{S}  \delta u_j \sigma_{ij} N_i \, dA  &= \int_{S_u}  \delta u_j \sigma_{ij} N_i \, dA + \int_{S_{\sigma}}  \delta u_j \sigma_{ij} N_i \, dA 
    \quad\quad\quad  \delta u_j = 0 \quad su \quad S_u\\
    & = \int_{S_{\sigma}}  \delta u_j \sigma_{ij} N_i \, dA 
\end{align*}

Sul contorno libero applico le forze esterne note $\mathbf{T}$.
\begin{equation*}
    \int_{S_{\sigma}}  \delta u_j \sigma_{ij} N_i \, dA  \, =\,   \int_{S_{\sigma}}  \delta u_j T_j \, dA 
\end{equation*}

Finalmente:
\begin{align*}
    & \int_{S_{\sigma}}  \delta u_j T_j \, dA  \, +\,\int_{\Omega} \delta u_{j}  b_j  \, dV \, = \, \int_{\Omega} \delta u_{j/i}  \sigma_{ij}  \, dV   \\
    &  \delta L_{totale} \,= \, \delta L_{deformazione}
\end{align*}

Ovvero il lavoro virtuale compiuto dalle forze esterne è uguale al lavoro virtuale di deformazione.




