\section{Risultante e momento risultante del flusso di taglio in un pannello}

Consideriamo il pannello $AB$, di forma generica sufficientemente regolare, poniamo $A$ come centro del sistema di riferimento, la componenti della forza risultante sono:

\begin{align*}
    R_x \,&=\, \int_A^B \,q\,dl\, cos \theta  \,=\,q \int_A^B \,dl\, cos \theta  \,=\,q \int_A^B \,dx \,=\ q \,x_B\\
      R_y \,&=\, \int_A^B \,q\,dl\, sin \theta  \,=\,q \int_A^B \,dl\, sin \theta  \,=\,q \int_A^B \,dy \,=\ q \,y_B
\end{align*}

Quindi ad esempio per un pannello verticale: $R_x = 0$, $R_y = ql$. La risultante dipende solo dai punti iniziali e finali, un pannello la cui linea media non è una funzione biettiva avrà parti non collaboranti. Un pannello chiuso ha risultanti nulle.\\
Per trovare il momento risultante prendiamo ad esempio un pannello ad arco, avente entrambi gli estremi sull'asse $x$. Le risultanti delle forze sono $R_y = 0$, $R_x = ql$. Per avere il braccio di $R_x$ è necessario capire a quale quota $d$ è applicata, scriviamo quindi l'equivalenza dei momenti. Il momento del flusso $   q$  attorno rispetto un polo $O$, con $C$ come punto arbitrario sul pannello:
\begin{align*}
    dm \,=\, qdl \,\cdot \, \bar{OC}\,=\,q\,2\, d\Omega_O\\
    M\,=\, \int_A^B \,dm\,=\,  \int_A^B \,q\,2\,d\Omega_O\,=\,2q\,\Omega_O
\end{align*}
dove $\Omega_O$ è l'area spazzata dal raggio vettore che va dal polo al pannello, fra un estremo e l'altro.\\
Dunque risulta:
\begin{equation*}
    \begin{cases}
        M=Rd\\
        M = 2q\,\Omega_O
    \end{cases}
    \quad\quad \rightarrow\quad\quad
    Rd= 2q\,\Omega_O
    \quad\quad \rightarrow\quad\quad
    q\,\bar{AB}\,d\,=\,2q\,\Omega_O
    \quad\quad \rightarrow\quad\quad
    d \,=\, \frac{2 \Omega_A}{\bar{AB}}
\end{equation*}

Il momento generato dal flusso di taglio per un pannello chiuso, rispetto a un polo interno risulta $M = 2q\,\Omega_O$, dove $\Omega_O$ è l'area interna, indipendente dal polo scelto. Il momento rispetto a un polo esterno è formalmente lo stesso, in quanto l'area esterna al pannello è percorsa due volte con versi opposti.

