\section{Stato di sforzo e deformazione piana.}

Consideriamo un corpo piano, le cui prime due dimensioni sono molto maggiori della terza.
\subsubsection*{Stato di sforzo piano}
Considero diverse da zero le sole componenti del tensore di Cauchy che giacciono nel piano $x,y$, ovvero:

\begin{align*}
         \sigma_{xx}\,\neq\,0\qquad\sigma_{yy}\,\neq\,0\qquad\sigma_{xy}\,\neq\,0
    \end{align*}


Lo stato di sforzo, confrontato a quello del modello di trave diventa:
    \begin{equation*}
     \left\{\sigma\right\}^{\text{piano}} = \left\{
    \begin{array}{c}
    \sigma_{xx} \\ \sigma_{yy} \\ \sigma_{xy}
    \end{array}
    \right\}
    \quad\quad\quad\quad
    \left\{\sigma\right\}^{\text{trave}} = \left\{
    \begin{array}{c}
    \sigma_{zz} \\ \sigma_{zx} \\ \sigma_{zy}
    \end{array}
    \right\}
 \end{equation*}

 Le componenti del tensore delle piccole deformazioni sono tutte diverse da zero. Consideriamo però significative solo le componenti coniugate agli elementi non nulli del tensore degli sforzi.\\ Se il materiale è isotropo stiamo ignorando solo $\varepsilon_{zz}\neq0$ perché gli scorrimenti $\gamma_{zx}=0$ e $\gamma_{zy}=0$ sono effettivamente nulli.

 Cerchiamo un legame costitutivo $\left\{\varepsilon\right\}=[C]\left\{\sigma\right\}$ considerando inizialmente tutte le componenti:

 \begin{equation*}
\left\{
\begin{array}{c}
\varepsilon_{xx} \\
\varepsilon_{yy} \\
\varepsilon_{zz} \\
\gamma_{yz} \\
\gamma_{xz} \\
\gamma_{xy}
\end{array}
\right\}
=
\left[
\begin{array}{cccccc}
\displaystyle \frac{1}{E} & \displaystyle -\frac{\nu}{E} & \displaystyle -\frac{\nu}{E} & & & \\
\displaystyle -\frac{\nu}{E} & \displaystyle \frac{1}{E} & \displaystyle -\frac{\nu}{E} & & \large{0} & \\
\displaystyle -\frac{\nu}{E} & \displaystyle -\frac{\nu}{E} & \displaystyle \frac{1}{E} & & & \\
& & & \displaystyle \frac{1}{G} & & \\
& \large{0} & & & \displaystyle \frac{1}{G} & \\
& & & & & \displaystyle \frac{1}{G}
\end{array}
\right]
\left\{
\begin{array}{c}
\sigma_{xx} \\
\sigma_{yy} \\
\sigma_{zz} \\
\sigma_{yz} \\
\sigma_{xz} \\
\sigma_{xy}
\end{array}
\right\}
\end{equation*}

Per lo stato di sforzo piano cancello le righe e le colonne corrispondenti agli elementi nulli.

\begin{equation*}
\left\{
\begin{array}{c}
\varepsilon_{xx} \\
\varepsilon_{yy} \\
\gamma_{xy}
\end{array}
\right\}
=
\left[
\begin{array}{ccc}
\displaystyle \frac{1}{E} & \displaystyle -\frac{\nu}{E} & \large{0}\\
\displaystyle -\frac{\nu}{E} & \displaystyle \frac{1}{E} & \large{0} \\
\large{0}& \large{0} & \displaystyle \frac{1}{G}
\end{array}
\right]
\left\{
\begin{array}{c}
\sigma_{xx} \\
\sigma_{yy} \\
\sigma_{xy}
\end{array}
\right\}
\end{equation*}

La matrice è invertibile per cui vale:
\begin{equation*}
    \left\{\sigma\right\}=[D] \left\{\varepsilon\right\}
\end{equation*}
dove $[D] = [C]^{-1}$
\begin{equation*}
\left\{
\begin{array}{c}
\sigma_{xx} \\
\sigma_{yy} \\
\sigma_{xy}
\end{array}
\right\}
=
\left[
\begin{array}{ccc}
\displaystyle \frac{E}{1-\nu^2} & \displaystyle \frac{\nu E}{1-\nu^2} & \large{0}\\
\displaystyle \frac{\nu E}{1-\nu^2} & \displaystyle \frac{E}{1-\nu^2} & \large{0} \\
\large{0}& \large{0} & G
\end{array}
\right]
\left\{
\begin{array}{c}
\varepsilon_{xx} \\
\varepsilon_{yy} \\
\gamma_{xy}
\end{array}
\right\}
\end{equation*}

\subsubsection*{Stato di deformazione piana}
Considero diverse da zero solo le componenti del tensore delle piccole deformazioni che giacciono sul piano $x,y$. Deformazioni nulle corrispondono ad impedire degli sforzi, per questo devo imporle nulle attraverso dei vincoli.

Tutte le componenti di $\boldsymbol{\sigma} $ sono diverse da zero, considero però solo le componenti che compiono lavoro, per cui:


    \begin{equation*}
     \left\{\varepsilon\right\} = \left\{
    \begin{array}{c}
    \varepsilon_{xx} \\ \varepsilon_{yy} \\ \varepsilon_{xy}
    \end{array}
    \right\}
    \quad\quad\quad\quad
     \left\{\sigma\right\} = \left\{
    \begin{array}{c}
    \sigma_{xx} \\ \sigma_{yy} \\ \sigma_{xy}
    \end{array}
    \right\}
 \end{equation*}
Considero il legame elastico $ \left\{\sigma\right\}=[D] \left\{\varepsilon\right\}$, con sei gradi di libertà.
\begin{equation*}
\left\{
\begin{array}{c}
\varepsilon_{xx} \\
\varepsilon_{yy} \\
\varepsilon_{zz} \\
\gamma_{yz} \\
\gamma_{xz} \\
\gamma_{xy}
\end{array}
\right\}
=
\frac{1}{E}
\begin{bmatrix}
1 & -\nu & -\nu & 0 & 0 & 0 \\
-\nu & 1 & -\nu & 0 & 0 & 0 \\
-\nu & -\nu & 1 & 0 & 0 & 0 \\
0 & 0 & 0 & 2(1+\nu) & 0 & 0 \\
0 & 0 & 0 & 0 & 2(1+\nu) & 0 \\
0 & 0 & 0 & 0 & 0 & 2(1+\nu)
\end{bmatrix}
\left\{
\begin{array}{c}
\sigma_{xx} \\
\sigma_{yy} \\
\sigma_{zz} \\
\sigma_{yz} \\
\sigma_{xz} \\
\sigma_{xy}
\end{array}
\right\}
\end{equation*}

Impongo i vincoli sulle deformazioni nulle fuori dal piano. Da questi vincoli $\sigma_{zz}\neq0$, però risulta dipendente da $\sigma_{xx}$ e $\sigma_{yy}$, quindi lo riscrivo in funzione di questi.

\begin{equation*}
\left\{
\begin{array}{c}
\sigma_{xx} \\
\sigma_{yy} \\
\sigma_{xy}
\end{array}
\right\}
=
\left[
\begin{array}{ccc}
\displaystyle 1-\nu & \displaystyle \nu & 0\\
\displaystyle \nu & 1-\nu & \large{0} \\
0& 0 & \frac{1-2\nu}{2}
\end{array}
\right]\,
\frac{E}{(1+\nu)(1-2\nu)}\,
\left\{
\begin{array}{c}
\varepsilon_{xx} \\
\varepsilon_{yy} \\
\gamma_{xy}
\end{array}
\right\}
\end{equation*}

E il suo inverso:

\begin{equation*}
\left\{
\begin{array}{c}
\varepsilon_{xx} \\
\varepsilon_{yy} \\
\gamma_{xy}
\end{array}
\right\}
=
\left[
\begin{array}{ccc}
\displaystyle 1-\nu & \displaystyle -\nu & 0\\
\displaystyle -\nu & 1-\nu & \large{0} \\
0& 0 & 2
\end{array}
\right]\,
\frac{1+\nu}{E}\,
\left\{
\begin{array}{c}
\sigma_{xx} \\
\sigma_{yy} \\
\sigma_{xy}
\end{array}
\right\}
\end{equation*}